% Use the following line _only_ if you're still using LaTeX 2.09.
%\documentstyle[icml2014,epsf,natbib]{article}
% If you rely on Latex2e packages, like most moden people use this:
\documentclass{article}

% use Times
\usepackage{times}
% For figures
\usepackage{graphicx} % more modern
%\usepackage{epsfig} % less modern
\usepackage{subfigure}

% For citations
\usepackage{natbib}

% For algorithms
\usepackage{algorithm}
\usepackage{algorithmic}

% As of 2011, we use the hyperref package to produce hyperlinks in the
% resulting PDF.  If this breaks your system, please commend out the
% following usepackage line and replace \usepackage{icml2014} with
% \usepackage[nohyperref]{icml2014} above.
\usepackage{hyperref}

% Packages hyperref and algorithmic misbehave sometimes.  We can fix
% this with the following command.
\newcommand{\theHalgorithm}{\arabic{algorithm}}

% Employ the following version of the ``usepackage'' statement for
% submitting the draft version of the paper for review.  This will set
% the note in the first column to ``Under review.  Do not distribute.''
\usepackage{icml2014}
% Employ this version of the ``usepackage'' statement after the paper has
% been accepted, when creating the final version.  This will set the
% note in the first column to ``Proceedings of the...''
%\usepackage[accepted]{icml2014}


% other packages
\usepackage{../Definitions}
\newcommand{\hmu}{\widehat{\mu}}
\newcommand{\hCcal}{\widehat{\Ccal}}
\newcommand\tl{\tilde}

\newcommand{\sk}[1]{\noindent{\textcolor{cyan}{\{{\bf SK:} \em #1\}}}}
\newcommand{\djh}[1]{\noindent{\textcolor{blue}{\textbf{\#\#\# DJH:} \textsf{#1} \#\#\#}}}
\newcommand{\aacomment}[1]{\noindent{\textcolor{blue}{\textbf{\#\#\# AA:} \textsf{#1} \#\#\#}}}
\newcommand{\rg}[1]{\noindent{\textcolor{blue}{\textbf{\#\#\# RG:} \textsf{#1} \#\#\#}}}
\DeclareMathOperator{\Ber}{Ber}

\def\tcr{\textcolor{red}}
\def\tcb{\textcolor{blue}}
\def\tha{{\mbox{\tiny th}}}
\def\Ibb{{\mathbb I}}

\DeclareMathOperator{\thres}{Thres}
\DeclareMathOperator{\Var}{Var}
\DeclareMathOperator\Diag{Diag}

\DeclareMathOperator{\polylog}{polylog}
% \DeclareMathOperator*{\argmin}{arg\,min}
% \DeclareMathOperator*{\argmax}{arg\,max}
\newcommand\E{\mathbb{E}}
\newcommand\R{\mathbb{R}}
\renewcommand\t{{\scriptscriptstyle\top}}
\newcommand\myvec[1]{\underline{#1}}
\newcommand\prior[1]{\ensuremath{\myvec{\pi_{#1}}}}
\newcommand\com[1]{\ensuremath{\myvec{o_{#1}}}}
\newcommand\e[1]{\ensuremath{\myvec{e_{#1}}}}
%\newcommand\ind[1]{\ensuremath{\mathds{1}_{\{#1\}}}}
\renewcommand\inner[1]{\ensuremath{\left<#1\right>}}
\newcommand\norm[1]{\left\|#1\right\|}
%\newcommand\bigO{\mathbb{O}}
\newcommand\bigO{O}
\newcommand\tlO{\tilde{\bigO}}
\def\t{{\scriptscriptstyle\top}}
\def\tl{\widetilde}
\def\h{\widehat}
\def\halpha{\widehat{\alpha}}
\def\tT{\tilde{T}}
\newcommand\dotp[1]{\langle #1 \rangle}
\def\Sig{\varSigma}
\def\tree{\mathcal{T}}
\def\hp{\hphantom}
\def\eps{\epsilon}
\def\veps{\varepsilon}
\def\simiid{{\overset{iid}{\sim}}}
\def\Pbb{{\mathbb P}}
\DeclareMathOperator{\poly}{poly}
\DeclareMathOperator{\Gammadist}{GammaCDF}
\def\Ac{{\cal A}}
\def\Bc{{\cal B}}
\def\bfd{{\mathbf d}}
\def\bfm{{\mathbf m}}
\def\Cc{{\cal C}}
\DeclareMathOperator{\Span}{span}
\renewcommand\th[1]{\ensuremath{\theta_{#1}}}
\newcommand\hth[1]{\ensuremath{\hat{\theta}_{#1}}}
\newcommand\teps{\ensuremath{\tilde{\eps}}}
\newcommand\hv{\ensuremath{\hat{v}}}
\newcommand\hlambda{\ensuremath{\hat{\lambda}}}
\newcommand\lambdamin{\ensuremath{\lambda_{\min}}}
\newcommand\lambdamax{\ensuremath{\lambda_{\max}}}
\newcommand\deflate{\mathcal{E}}
 \newcommand\tlambdamin{\ensuremath{\tilde{\lambda}_{\min}}}

\DeclareMathOperator{\eqa}{\overset{(a)}{=}}


\newcommand{\red}[1]{\textcolor{red}{#1}}
%\DeclareMathOperator{\diag}{diag}
\DeclareMathOperator{\range}{range}
%\DeclareMathOperator{\rank}{rank}
\DeclareMathOperator{\sign}{sign}
\DeclareMathOperator{\Pairs}{Pairs}
\DeclareMathOperator{\Triples}{T}

\newcommand\Dir{\operatorname{Dir}}

% environments
%\newtheorem{theorem}{Theorem}[section]
%\newtheorem*{namedtheorem}{\theoremname}
%\newcommand{\theoremname}{testing}
%\newenvironment{named}[1]{ \renewcommand{\theoremname}{#1} \begin{namedtheorem}} {\end{namedtheorem}}
%\newtheorem{thm}[theorem]{Theorem}
%\newtheorem{lemma}[theorem]{Lemma}
%\newtheorem{lem}[theorem]{Lemma}
%\newtheorem*{claim}{Claim}
%\newtheorem{clm}[theorem]{Claim}
%\newtheorem{proposition}[theorem]{Proposition}
%\newtheorem{prop}[theorem]{Proposition}
%\newtheorem{fact}[theorem]{Fact}
%\newtheorem{res}{Restriction}
%\newtheorem{observation}{Observation}
%\newtheorem{corollary}[theorem]{Corollary}
%\newtheorem{cor}[theorem]{Corollary}
%\theoremstyle{definition}
%\newtheorem{definition}[theorem]{Definition}
%\newtheorem{defn}[theorem]{Definition}

\newcommand{\bp}{\begin{psfrags}}
\newcommand{\ep}{\end{psfrags}}
\newcommand{\bprfof}{\begin{proof_of}}
\newcommand{\eprfof}{\end{proof_of}}
\newcommand{\bprf}{\begin{myproof}}
\newcommand{\eprf}{\end{myproof}}

\newenvironment{myproof}{\noindent{\em Proof:} \hspace*{1em}}{
    \hspace*{\fill} $\Box$ }
\newenvironment{proof_of}[1]{\noindent {\em Proof of #1: }}{\hspace*{\fill} $\Box$ }

\def\viz{{viz.,\ \/}}
\def\Ebb{{\mathbb E}}
\def\nn{\nonumber}
\def\beq{\begin{equation}}
\def\eeq{\end{equation}\noindent}
\def\beqn{\begin{eqnarray}}
\def\eeqn{\end{eqnarray} \noindent}
\def\beqnn{  \begin{eqnarray*}}
\def\eeqnn{\end{eqnarray*}  \noindent}
\def\bcase{  \begin{numcases}}
\def\ecase{\end{numcases}   \noindent}

\usepackage{tikz}
\usetikzlibrary{calc}
\usepackage{exscale,relsize}
\usepackage{enumitem}

\usepackage{color}
\newcommand{\Note}[1]{{\color{red}{\bf\sf [note: #1]}}}

% The \icmltitle you define below is probably too long as a header.
% Therefore, a short form for the running title is supplied here:
\icmltitlerunning{Nonparametric Estimation of Multi-view Latent Variable Models}

\begin{document}

\twocolumn[
\icmltitle{Nonparametric Estimation of Multi-View Latent Variable Models}

% It is OKAY to include author information, even for blind
% submissions: the style file will automatically remove it for you
% unless you've provided the [accepted] option to the icml2014
% package.
\icmlauthor{Le Song, Anima, B. Dai and B. Xie}{email@yourdomain.edu}
\icmladdress{Your Fantastic Institute,
            314159 Pi St., Palo Alto, CA 94306 USA}
\icmlauthor{Your CoAuthor's Name}{email@coauthordomain.edu}
\icmladdress{Their Fantastic Institute,
            27182 Exp St., Toronto, ON M6H 2T1 CANADA}

% You may provide any keywords that you
% find helpful for describing your paper; these are used to populate
% the "keywords" metadata in the PDF but will not be shown in the document
\icmlkeywords{Kernel Methods, Latent Variable Models, Spectral Algorithms}

\vskip 0.3in
]

\begin{abstract}
		Spectral methods have greatly advanced the estimation of latent variable models, generating a sequence of novel and efficient algorithms with strong theoretical guarantees. However, current spectral algorithms are largely restricted to mixture of discrete or Gaussian distributions. In this paper, we propose a kernel method for estimating the parameters of multi-view latent variable models, allowing each mixture component to be nonparametric.
		The key idea of the method is to embed the joint distribution of a multi-view latent variable into a reproducing kernel Hilbert space, and then the latent parameters are recovered using a robust tensor power method. Furthermore, we provide theoretical guarantee for the proposed method and show that it compares favorably to EM algorithm and other existing spectral algorithms in our experiments.
\end{abstract}

\section{Introduction} \label{sec:intro}

\setlength{\abovedisplayskip}{4pt}
\setlength{\abovedisplayshortskip}{1pt}
\setlength{\belowdisplayskip}{4pt}
\setlength{\belowdisplayshortskip}{1pt}
\setlength{\jot}{3pt}

%\setlength{\floatsep}{1ex}
\setlength{\textfloatsep}{3ex}

Latent variable models have been used to address various machine learning
problems, ranging from modeling temporal dynamics, to text document analysis and to social network analysis~\citep{RabJua86,Clark90,HofRafHan02,BleNgJor03}.
Recently, there is a surge of interest in designing spectral algorithms for estimating the parameters of latent variable models~\citep{HsuKakZha09,ParSonXin11,SonParXin11,FosRodUng12,more on parameter estimation}. Compared to the Expectation-Maximization (EM) algorithm \citep{DemLaiRub77} traditionally used for the same task, spectral algorithms are better in terms of their computational efficiency and provable guarantees.

However, current spectral algorithms are largely restricted to mixture
of discrete or Gaussian distributions. When the mixture components are distributions other than these standard distributions, the theoretical guarantees for these algorithms no longer applies, and the performance of these algorithms can be very poor in experiments. A nonparametric estimation of the parameters of latent variables with provable guarantee is still missing from the literature.

In this paper, we propose a kernel method for estimating the parameters of multi-view latent variable models where the mixture components can be nonparametric. The key idea of the method is to embed the joint distribution of such a model into a reproducing kernel Hilbert space, and make use of the low rank structure of the embedded distribution (or covariance operators) to design the algorithm. The key computation of the algorithm involves a kernel singular value decomposition of the two-view covariance operator, followed by a tensor power method on the three-view covariance operator. These standard linear algebra operations makes the algorithm very efficient and easy to code.

The kernel algorithm proposed in this paper is more general than previous spectral algorithms which work only for distributions with very specific assumption. When we use delta kernel, our algorithm recovers the spectral algorithm for discrete mixture components. When we use universal kernels, such as Gaussian RBF kernel, our algorithm can recover Gaussian mixture components as well as mixture components with other distributions. In a sense, our work also provides a unifying framework for previous spectral algorithms. We will provide theoretical guarantees for our general kernel algorithm, and these guarantees will automatically carry over to those special cases.

Previously, kernel methods have also been applied to learning latent variable models. However, none of the previous work explicitly recovers the actual parameters of the models~\cite{xxx}. Most of them estimate an (unknown) invertible  transformation of the latent parameters, and it is not clear how one can recover the actual parameters based on these estimates. Furthermore, these works focused on predictive task: recover the marginal distribution of the observed variables by making use of the low rank structure of the latent variable models. It is significantly more challenging to design kernel algorithms for actual parameter recovery and analyze theoretical properties of these algorithms.

Last we will experimentally compare our kernel algorithm to the EM algorithm and previous spectral algorithms. We show that when the model assumptions are correct for the EM algorithm and previous spectral algorithms, our algorithm converges in terms of estimation error to these competitors. In the opposite cases when the model assumptions are incorrect, our algorithm is able to adapt to the nonparametric mixture components and beating alternatives by a very large margin.

%%%%%%%%%%%%%%%%%%%%%%%%%%%%%%%%%%%%%%%%%%%%%%%%%%%%%%%%%%%%%%%%%%%%%%%%%%%%%%%%%%%%%%%%%%%%%%
\section{Notation}
%%%%%%%%%%%%%%%%%%%%%%%%%%%%%%%%%%%%%%%%%%%%%%%%%%%%%%%%%%%%%%%%%%%%%%%%%%%%%%%%%%%%%%%%%%%%%%

We will denote by $X$ a random variable with domain $\Xcal$,
and refer to instantiations of $X$ by the lower case character, $x$.
We endow $\Xcal$ with some $\sigma$-algebra $\Ascr$ and denote a distributions (with respect to $\Ascr$) on $\Xcal$ by $\PP(X)$. We will also deal with multiple random variables, $X_1, X_2, \ldots, X_{\ell}$, with joint distribution $\PP(X_1,X_2,\ldots,X_{\ell})$. For simplicity of notation, we assume that the domains of all $X_t, t \in [\ell]$ are the same, but the methodology applies to the cases where they have different domains. Furthermore, we denote by $H$ a hidden variables with domain $\Hcal$ and distribution $\PP(H)$.

A \emph{reproducing kernel Hilbert space (RKHS)} $\Fcal$ on $\Xcal$ with a kernel $k(x,x')$ is a Hilbert space of
functions $f(\cdot):\Xcal \mapsto \RR$ with inner product $\inner{\cdot}{\cdot}_{\Fcal}$. Its element $k(x,\cdot)$ satisfies the reproducing property:
$\inner{f(\cdot)}{k(x, \cdot)}_{\Fcal} = f(x)$, and consequently, $\inner{k(x,\cdot)}{k(x', \cdot)}_{\Fcal} = k(x,x')$,
meaning that we can view the evaluation of a function $f$ at any point $x\in\Xcal$ as an inner product. Alternatively, $k(x,\cdot)$ can  be viewed as an implicit feature map $\phi(x)$ where $k(x,x')=\inner{\phi(x)}{\phi(x')}_{\Fcal}$.
Popular kernel functions on $\RR^n$ include the Gaussian RBF kernel $k(x,x') = \exp(-s
  \nbr{x-x'}^2)$ and the Laplace kernel $\exp(-s \|x - x'\|)$. Kernel functions have also been defined on
graphs, time series, dynamical systems, images and other structured
objects \, \cite{SchTsuVer04}. Thus the methodology presented below can readily be generalized to a diverse range of data types as long as kernel functions are defined.
% Similarly, we denote by $\Gcal$ an RKHS on $\Hcal$ with kernel $l(h,h')$, and by $\psi(h)$ the corresponding feature map.

% Furthermore, we let $h \in [k]$ be a discrete random variable with $P(h =
% j) = \pi_j$ for all $j \in [k]$.
%
% We denote by $X$ a random variable with domain $\Xcal$ and distribution $P(X)$, and refer to instantiations of $X$ by the lower case character, $x$.
% We will focus on continuous domains, and denote the corresponding density by $p(X)$. We will also deal with multiple random variables, $X_1, X_2, \ldots, X_{\ell}$, with joint distribution $P(X_1,X_2,\ldots,X_{\ell})$. For simplicity of notation, we assume that the domain of all $X_t, t \in [\ell]$ also be $\Xcal$, but the methodology applies to the cases where they have different domains. Furthermore, we let $H \in [k]$ be a discrete random variable with $P(H =
% j) = \pi_j$ for all $j \in [k]$.



\section{Kernel Embedding of Distributions}
\label{sec:embedding}

We begin by providing an overview of kernel embeddings of distributions, which are \emph{implicit} mappings of distributions into potentially \emph{infinite} dimensional RKHS.\footnote{By ``implicit'', we mean that we do not need to explicitly construct the feature spaces, and the actual computations boil down to kernel matrix operations.} The kernel embedding approach represents a distribution by an element in the RKHS associated with a kernel function \, \cite{SmoGreSonSch07,SriGreFukLanetal08},
\begin{align}
  \mu_{X} \, := \, \EE_{X} \sbr{\phi(X)} \, = \, \int_{\Xcal} \phi(x) \, \PP(dx),  \label{eq:embedding}
\end{align}
where the distribution is mapped to its expected feature map,~\ie,~to a point in a potentially infinite-dimensional and implicit feature space.
 The kernel embedding $\mu_{X}$ has the property that the expectation of any RKHS function $f$ can be evaluated as an inner product in $\Fcal$,
$
  \EE_{X} [f(X)] = \inner{\mu_{X}}{f}_{\Fcal}, \, \forall f\in\Fcal.
$

Kernel embeddings can be readily generalized to joint distributions of two or more variables using tensor product feature maps. For instance, we can embed a joint distribution of two variables $X_1$ and $X_2$ into a tensor product feature space $\Fcal\times \Fcal$ by
\begin{align}
    \Ccal_{X_1X_2} \, &:= \, \EE_{X_1X_2}[\phi(X_1)\otimes \phi(X_2)] \\
    \, &= \, \int_{\Xcal \times \Xcal} \phi(x_1) \otimes \phi(x_2) \, \PP(dx_1 \times dx_2)
\end{align}
where the reproducing kernel for the tensor product features satisfies
$
	\inner{\phi(x_1)\otimes \phi(x_2) }{\phi(x_1')\otimes \phi(x_2') }_{\Fcal\times \Fcal} \, = \,  k(x_1,x_1')\,k(x_2,x_2').
$
By analogy, we can also define $\Ccal_{X_1X_2X_3} := \EE_{X_1X_2X_3}[\phi(X_1)\otimes\phi(X_2)\otimes \phi(X_3)]$.

Kernel embedding of distributions has rich representational power. The mapping is injective for characteristic kernels~\cite{SriGreFukLanetal08}. That is, if two distributions, $\PP(X)$ and $\QQ(X)$, are different, they will be mapped to two distinct points in the RKHS. For domain $\RR^d$, many commonly used kernels are characteristic, such as the Gaussian RBF kernel and Laplace kernel.
This injective property of kernel embeddings has been exploited to design  state-of-the-art two-sample tests~\cite{GreBorRasSchetal12} and  independence tests~\cite{GreFukTeoSonetal08}.

{\bf Example 1.} The probability vector of a discrete variable $X \in [n]$, and the joint probability table of two discrete variables $X_1 \in [n]$ and $X_2 \in [n]$, are both kernel embeddings. To see this, let the kernel be the Kronecker delta kernel $k(x,x') = \delta(x,x')$ whose feature map $\phi(x)$ is the standard basis of $e_{x}$ in $\RR^n$. The $x$-th dimension of $e_{x}$ is 1 and 0 otherwise. Then
\begin{align}
    \mu_X
		& = \rbr{
      \begin{array}{ccc}
         \PP(x = 1) &
         \hdots &
         \PP(x = n)
       \end{array}
    }^\top, \nonumber \\
		\Ccal_{X_1X_2}
		&=
		\rbr{
        \begin{array}{c}
            \cr
            \PP(x_1=s,x_2=t) \cr
						\cr
        \end{array}
    }_{s,t \in [n]}. \nonumber
% 	\label{eq:jointprobabilitytable}
\end{align}

\subsection{Kernel Embedding as Multi-Linear Operator}

The joint embeddings can also be viewed as an uncentered covariance operator $\Ccal_{X_1X_2}:\Fcal\mapsto \Fcal$ by the standard equivalence between a tensor product feature and a linear map.
That is, given two functions $f_1,f_2\in\Fcal$, their covariance can be computed by
$
    \EE_{X_1X_2}[f_1(X_1)f_2(X_2)]=\inner{f_1}{\Ccal_{X_1X_2} f_2}_{\Fcal}
$
, or equivalently
$
\inner{f_1\otimes f_2}{\Ccal_{X_1X_2}}_{\Fcal\times\Fcal},
$
where in the former we view $\Ccal_{XY}$ as an operator while in the latter we view it as an element in tensor product feature space.
By analogy, $\Ccal_{X_1X_2X_3}$ can be regarded as a multi-linear operator from $\Fcal\times\Fcal\times\Fcal$ to $\RR$.
It will be clear from the context whether we use $\Ccal_{XY}$ as an operator between two spaces or as an element from a tensor product feature space. For generic introduction to tensor and tensor notation, please see~\cite{KolBad09}.

The operator $\Ccal_{X_1X_2X_3}$ (with shorthand $\Ccal_{X_{1:3}}$) is linear in each argument (mode) when fixing other arguments. Furthermore, the application of the operator to a set of elements $\cbr{f_1,f_2,f_3 \in \Fcal}$ can be defined using the inner product from the tensor product feature space,~\ie,
\begin{align}
	\Ccal_{X_{1:3}} \times_1 f_1 \times_2 \times_3 f_3
	&:= 	\inner{\Ccal_{X_{1:3}}}{\;f_1\otimes f_2\otimes f_3}_{\Fcal^3} \nonumber \\
	&=\EE_{X_1X_2X_3}\sbr{\prod_{i \in [3]} \inner{\phi(X_i)}{~f_i}_{\Fcal}}, \nonumber
\end{align}
where $\times_i$ means applying $f_i$ to the $i$-th argument of $\Ccal_{X_{1:3}}$. Furthermore, we can define the Hilbert-Schmidt norm $\nbr{\cdot}_{}$ of $\Ccal_{X_{1:3}}$ as
\[
 \nbr{\Ccal_{X_{1:3}}}_{}^2 = \sum_{i_1 = 1}^\infty \sum_{i_2 = 1}^\infty \sum_{i_3 = 1}^\infty \rbr{\Ccal_{X_{1:3}} \times_1 u_{i_1} \times_2 u_{i_2} \times_3 u_{i_3}}^2
\]
using three collections of orthonormal bases $\cbr{u_{i_1}}_{i_1=1}^\infty$, $\cbr{u_{i_2}}_{i_2=1}^\infty$, and $\cbr{u_{i_3}}_{i_3=1}^\infty$. We can also define the inner product for the space of such operator that $\nbr{\Ccal_{X_{1:3}}}_{}<\infty$
\begin{align}
	\inner{\Ccal_{X_{1:3}}}{~\widetilde{\Ccal}_{X_{1:3}}}_{}  &=  \sum_{i_1 = 1}^\infty \sum_{i_2 = 1}^\infty \sum_{i_\ell = 1}^\infty \rbr{\Ccal_{X_{1:\ell}} \times_1 u_{i_1} \times_2 u_{i_2} \times_3 u_{i_3}} \nonumber \\
	& \cdot\ (\widetilde{\Ccal}_{X_{1:\ell}} \times_1 u_{i_1} \times_2 \ldots \times_\ell u_{i_\ell}). \nonumber
\end{align}
% When $\Ccal_{X_{1:\ell}}$ has the form of $\EE_{X_{1:\ell}}\sbr{\phi(X_1)\otimes\ldots\otimes \phi(X_\ell)}$, the above inner product reduces to $\EE_{X_{1:\ell}}[\widetilde{\Ccal}_{X_{1:\ell}} \times_1 \phi(X_1) \times_2 \ldots \times_\ell \phi(X_\ell)]$.
%
The joint embedding, $\Ccal_{X_1 X_2}$, is a 2nd order tensor, and we can essentially use notation and operations for matrices. For instance, we can perform singular value decomposition
\[
    \Ccal_{X_1X_2} = \sum_{i=1}^{\infty} \sigma_i \cdot u_{i_1} \otimes u_{i_2}
\]
where $\sigma_i \in \RR$ are singular values ordered in nonincreasing manner, and $\cbr{u_{i_1}}_{i_1=1}^{\infty} \subset \Fcal, \cbr{u_{i_2}}_{i_2=1}^{\infty} \subset \Fcal$ are singular vectors and orthonormal bases. The rank of $\Ccal_{X_1X_2}$ is the smallest $k$ such that $\sigma_i = 0$ for $i > k$.

\subsection{Finite Sample Estimate}

While we rarely have access to the true underlying distribution, $\PP(X)$,
we can readily estimate its embedding using a finite sample average. Given a sample $\Dcal_{X} = \cbr{x^1, \ldots, x^m}$ of size $m$ drawn~\iid~from $\PP(X)$, the empirical kernel embedding is
\begin{align}
    \hmu_{X} &:= \frac{1}{m} \sum\nolimits_{i=1}^m \phi(x^i). \label{eq:empirical_embedding}
\end{align}
This empirical estimate converges to its population counterpart in RKHS norm, $\|\hmu_X - \mu_X \|_{\Fcal}$, with a rate of $O_p(m^{-\frac{1}{2}})$~\cite{SmoGreSonSch07}.
% We note that this rate is independent of the dimension of $X$, meaning that statistics based on kernel embeddings circumvent the curse of dimensionality.

The covariance operator can be estimated similarly using finite sample average.
Given $m$ pairs of training examples $\Dcal_{XY}=\cbr{(x_1^i,x_2^i)}_{i\in [m]}$ drawn~\iid~from $\PP(X_1,X_2)$,
\begin{align}
 \widehat \Ccal_{X_1X_2} =\frac{1}{m}\sum_{i=1}^m \phi(x_1^i) \otimes \phi(x_2^i). \label{eq:empirical_covariance}
\end{align}
Similarly, given sample from distribution $\PP(X_1,X_2,X_3)$, one can estimate $\widehat \Ccal_{X_{1:3}} = \frac{1}{m} \sum_{i = 1}^m \phi(x_1^i) \otimes \phi(x_2^i) \otimes \phi(x_3^i)$.

By virtue of the kernel trick, most of the computation required for subsequent statistical inference using kernel embeddings can be reduced to the Gram matrix manipulation. The entries in the Gram matrix $K$ correspond to the kernel value between data points $x^i$ and $x^j$,~\ie,~$K_{ij} = k(x^i,x^j)$, and therefore its size is determined by the number of data points in the sample. The size of the Gram matrix is in general much smaller than the dimension of the feature spaces (which can be infinite). This enables efficient nonparametric methods using the kernel embedding representation. If the sample size is large, the computation in kernel embedding methods may be expensive. In this case, a popular solution is to use a low-rank approximation of the Gram matrix, such as incomplete Cholesky factorization~\cite{FinSch01}, which is known to work very effectively in reducing computational cost of kernel methods, while maintaining the approximation accuracy.

\noindent {\bf Relation between kernel embedding and the density function.} Basically kernel embeddings maps the density to a function in the RKHS. See Steinwardt and Vert (need to say more).


%%%%%%%%%%%%%%%%%%%%%%%%%%%%%%%%%%%%%%%%%%%%%%%%%%%%%%%%%%%%%%%%%%%%%%%%%%%%%%%%%%%%%%%%%%%%%%
\section{Multi-view Latent Variable Models}
%%%%%%%%%%%%%%%%%%%%%%%%%%%%%%%%%%%%%%%%%%%%%%%%%%%%%%%%%%%%%%%%%%%%%%%%%%%%%%%%%%%%%%%%%%%%%%

Multi-view latent variable models studied in this paper are a
special class of Bayesian networks in which
\begin{itemize}[noitemsep,nolistsep]
  \item observed variables $X_1, X_2, \ldots, X_\ell$ are conditionally independent given a {\bf discrete} latent variable $H$, and
	\item the conditional distributions, $\PP(X_t|H)$, of the $X_t, t \in [\ell]$ given the hidden variable $H$ can be different.
\end{itemize}
The conditional independent structure of a multi-view latent variable model is illustrated in Figure~\ref{fig:graphical-model}(a), and many complicated graphical models, such as the hidden Markov model in Figure~\ref{fig:graphical-model}(b), can be reduced to a multi-view latent variable model.

\begin{figure}[t]
\begin{center}
\subfigure[Na\"ive Bayes model]{\label{fig:exchangeable}
\begin{tikzpicture}
 [
   scale=0.85,
   observed/.style={circle,minimum size=0.7cm,inner sep=0mm,draw=black,fill=black!20},
   hidden/.style={circle,minimum size=0.7cm,inner sep=0mm,draw=black},
 ]
 \node [hidden,name=h] at ($(0,0)$) {$H$};
 \node [observed,name=x1] at ($(-1.5,-1)$) {$X_1$};
 \node [observed,name=x2] at ($(-0.5,-1)$) {$X_2$};
 \node at ($(0.5,-1)$) {$\dotsb$};
 \node [observed,name=xl] at ($(1.5,-1)$) {$X_\ell$};
 \draw [->] (h) to (x1);
 \draw [->] (h) to (x2);
 \draw [->] (h) to (xl);
\end{tikzpicture}}
\hfil\subfigure[Hidden Markov model]{\label{fig:hmm}
\begin{tikzpicture}
 [
   scale=0.85,
   observed/.style={circle,minimum size=0.7cm,inner sep=0mm,draw=black,fill=black!20},
   hidden/.style={circle,minimum size=0.7cm,inner sep=0mm,draw=black},
 ]
 \node [hidden,name=h1] at ($(-1.2,0)$) {$H_1$};
 \node [hidden,name=h2] at ($(0,0)$) {$H_2$};
 \node [name=hd] at ($(1.2,0)$) {$\dotsb$};
 \node [hidden,name=hl] at ($(2.4,0)$) {$H_\ell$};
 \node [observed,name=x1] at ($(-1.2,-1)$) {$X_1$};
 \node [observed,name=x2] at ($(0,-1)$) {$X_2$};
 \node [observed,name=xl] at ($(2.4,-1)$) {$X_\ell$};
 \draw [->] (h1) to (h2);
 \draw [->] (h2) to (hd);
 \draw [->] (hd) to (hl);
 \draw [->] (h1) to (x1);
 \draw [->] (h2) to (x2);
 \draw [->] (hl) to (xl);
\end{tikzpicture}}
\end{center}
\vspace{-5mm}
\caption{Examples of multi-view latent variable models.
}
\label{fig:graphical-model}
\vspace{-3mm}
\end{figure}

\subsection{Conditional Embedding Operator}

For simplicity of exposition, we will focus on a simple model with three observed variables, \ie, $\ell=3$, and furthermore the conditional distributions, $\PP(X_t|H)$, are the same (or {\em exchangeable}) for $t \in [3]$. Our analysis can be easily extended to the general multi-view setting using a symmetrization techniques (see section~X for details).

Suppose $H\in[k]$, then we can embed each conditional distribution $\PP(X_t|h)$ corresponding to a particular value of $H=h$ into the RKHS as
\begin{align}
  \mu_{X_t|h} = \int_{\Xcal} \phi(x_t)\, \PP(dx_t | h).
\end{align}
If we vary the value of $H$, we will obtain the kernel embedding for different $\PP(X_t|h)$. Conceptually, we can collect these embeddings into a matrix (with potentially infinite number of rows)
\begin{align}
  \Ccal_{X_t|H} = \rbr{\mu_{X_t|h=1},\mu_{X_t|h=2},\ldots,\mu_{X_t|h=k}},
\end{align}
which is called the conditional embedding operator. If we use the standard basis $e_h$ in $\RR^k$ to represent each value of $h$, we can retrieve each $\mu_{X_t|h}$ from $\Ccal_{X_t|H}$ by
\begin{align}
  \mu_{X_t|h} = \Ccal_{X_t|H} e_h
\end{align}

\subsection{Factorized Kernel Embedding}

Then the distributions, $\PP(X_1,X_2)$ and $\PP(X_1,X_2,X_3)$, can be factorized respectively
\begin{align*}
	\PP(dx_1, dx_2) &= \int_{\Hcal}\, \PP(dx_1 | h )\, \PP(d x_2 | h)\, \PP(dh),~\text{and}\\
	\PP(dx_1, dx_2, dx_3) &= \int_{\Hcal}\, \PP(dx_1 | h )\, \PP(dx_2 | h)\, \PP(dx_3 | h)\, \PP(dh).
\end{align*}
Since we assume the hidden variable $H \in [k]$ is discrete,
we let $\pi_h:=\PP(h)$. Furthermore, if we apply Kronecker delta kernel $\delta(h,h')$ with feature map $e_h$, then the embeddings for $\PP(H)$
\begin{align*}
 \Ccal_{HH} &= \EE_H[e_H \otimes e_H] = \rbr{
  \begin{array}{ccc}
    \pi_1 & \ldots & 0 \cr
    \vdots & \ddots & \vdots \cr
    0 & \ldots & \pi_k
  \end{array}
 },
 ~\text{and}~\\
 \Ccal_{HHH} &= \EE_H[e_H \otimes e_H \otimes e_H]  \\
	&= \rbr{
  \begin{array}{c}
    \cr
    \pi_h\ \delta(h,h')\ \delta(h',h'') \cr
    \cr
  \end{array}
 }_{h,h',h''\in[k]}
\end{align*}
are diagonal tensors. Making use of $\Ccal_{HH}$ and $\Ccal_{HHH}$, and the factorization of the distributions $\PP(X_1,X_2)$ and $\PP(X_1,X_2,X_3)$, we obtain the factorization of their embeddings
respectively
\begin{align}
  &\Ccal_{X_1X_2} \nonumber \\
%   &= \int_{\Omega}\, \phi(x_1) \otimes \phi(x_2)\, \PP(dx_1 | h )\, \PP(dx_2 | h)\, \PP(dh) \\
  &= \int_{\Hcal} \rbr{\int_\Xcal \phi(x_1)\, \PP(dx_1|h)} \otimes \rbr{\int_\Xcal \phi(x_2)\, \PP(dx_2 |h)} \PP(dh) \nonumber \\
  &= \int_{\Hcal} \rbr{\Ccal_{X|H} e_h} \otimes \rbr{\Ccal_{X|H} e_h}\, \PP(dh) \nonumber \\
  &= \Ccal_{X|H}\, \rbr{\int_{\Hcal} e_h \otimes e_h\, \PP(dh)}\, \Ccal_{X|H}^\top \nonumber \\
  &= \Ccal_{X|H}\, \Ccal_{HH}\, \Ccal_{X|H}^\top, \label{eq:pair_factorization}\\
  &\Ccal_{X_1 X_2 X_3}
  = \Ccal_{HHH} \times_1 \Ccal_{X|H} \times_2 \Ccal_{X|H} \times_3 \Ccal_{X|H}. \label{eq:triple_factorization}
\end{align}

\subsection{Identifiability of the Parameters}

We note that $\Ccal_{X|H} = \rbr{\mu_{X|h=1},\ \mu_{X|h=2},\ \ldots,\ \mu_{X|h=k}}$, and the kernel embeddings for $\Ccal_{X_1 X_2}$ and $\Ccal_{X_1 X_2 X_3}$ can be alternatively written as
\begin{align}
	\Ccal_{X_1 X_2}
  & = \sum_{h \in [k]} \pi_h \cdot \mu_{X|h} \otimes \mu_{X|h}, \label{eq:joint2} \\
  \Ccal_{X_1 X_2 X_3}
  &= \sum_{h \in [k]} \pi_h \cdot \mu_{X|h} \otimes \mu_{X|h} \otimes \mu_{X|h} \label{eq:joint3}
\end{align}
Allman et al.~\cite{x} showed that, under mild conditions, a finite mixture of nonparametric product distributions is identifiable. The multi-view latent variable model in~\eq{eq:joint3} has the same form as a finite mixture of nonparametric product distribution, and therefore we can adapt Allman's results to the current setting.
\begin{theorem}
  Let $\PP(X_1,X_2,X_3)$ be a multi-view latent variable model, such that the conditional distributions $\cbr{\PP(X|h)}_{h \in [k]}$ are linearly independent. Then, the set of parameters $\cbr{\pi_h, \mu_{X|h}}_{h \in [k]}$ are identifiable from $\Ccal_{X_1 X_2 X_3}$, up to label swapping of the hidden variable $H$.
\end{theorem}

% \section{Whitening with Three Different Views}

% \begin{itemize}
% 	\item $\widetilde{\Ccal}_{12} = \Ccal_{12} \Ucal_2^\top$; $\widetilde{\Ccal}_{13} = \Ccal_{13} \Ucal_3^\top$; $\widetilde{\Ccal}_{23} = \Ucal_{2} \Ccal_{23} \Ucal_3^\top$
% 	\item $\widetilde{\Ccal}_{11} = \widetilde{\Ccal}_{12} (\widetilde{\Ccal}_{32})^{-1} \widetilde{\Ccal}_{31}^\top = \widetilde{\Ucal}_1 \widetilde{\Scal}_1 \widetilde{\Ucal}_1^\top$
% 	\item $\widetilde{\Tcal} = \Ccal_{123} \times_1 (\widetilde{\Ucal}_1 \widetilde{\Scal}_1^{-1/2})^\top  \times_2 (\widetilde{\Ucal}_2 \widetilde{\Scal}_2^{-1/2})^\top \times_3 (\widetilde{\Ucal}_3 \widetilde{\Scal}_3^{-1/2})^\top$
% \end{itemize}

\section{Kernel Algorithm}

Next we will design a kernel algorithm to recover the parameters, $\cbr{\pi_h, \mu_{X|h}}_{h \in [k]}$, of the multi-view latent variable model based on $\Ccal_{X_1 X_2}$ and $\Ccal_{X_1 X_2 X_3}$.

\Note{say something about how direct the replacement is.}

\subsection{Population Case}

We will first derive the algorithm for the population case as if we could access the true operator $\Ccal_{X_1 X_2}$ and $\Ccal_{X_1 X_2 X_3}$. Its finite sample counterpart will be presented in the next section. The algorithm can be thought of as a kernel generalization of the algorithm in~\cite{AnandkumarEtal:community12} using embedding representations. 

{\bf Step 1.} We perform eigen-decomposition of $\Ccal_{X_1 X_2}$, 
$$\Ccal_{X_1 X_2} = \sum_{i=1}^{\infty} \sigma_i \cdot u_i \otimes u_i$$
where the eigen-values are ordered in non-decreasing manner. 
According to the factorization in Eq.~\eq{eq:pair_factorization}, $\Ccal_{X_1 X_2}$ has rank $k$. 
Let the leading eigenvectors corresponding to the largest $k$ eigen-value be  $\Ucal_k:=(u_1,u_2,\ldots,u_k)$, and the eigen-value matrix be $S_k:=\diag(\sigma_1,\sigma_2,\ldots,\sigma_k)$. We define the whitening operator $\Wcal:= \Ucal_k S_k^{-1/2}$ which satisfies 
\begin{align*}
  \Wcal^\top \Ccal_{X_1X_2} \Wcal = (\Wcal^\top \Ccal_{X|H} \Ccal_{HH}^{1/2}) (\Ccal_{HH}^{1/2} \Ccal_{X|H}^\top \Wcal) = I, 
\end{align*}
and $M:=\Wcal^\top \Ccal_{X|H} \Ccal_{HH}^{1/2}$ is an orthogonal matrix. 
    
{\bf Step 2.} We apply the whiten operator to the 3rd order kernel embedding $\Ccal_{X_1 X_2 X_3}$
$$
  \Tcal := \Ccal_{X_1 X_2 X_3} \times_1 (\Wcal^\top) \times_2 (\Wcal^\top) \times_3 (\Wcal^\top). 
$$
According to the factorization in Eq.~\eq{eq:triple_factorization}, 
$$
  \Tcal = \Ccal_{HHH}^{-1/2} \times_1 M \times_2 M \times_3 M, 
$$
which is a tensor with orthogonal factors. Essentially, each column $v_i$ of $M$ is an eigenvector of the tensor $\Tcal$. 

{\bf Step 3.} We find the tensor eigenvectors $M$ for $\Tcal$ using tensor power method~\cite{AnandkumarEtal:community12}.

{\bf Step 4.} We recover the conditional embedding operator by undoing the whitening step
$$ 
  \Ccal_{X|H} = (\mu_{X|h=1},\mu_{X|h=1},\ldots,\mu_{X|h=k}) = (\Wcal)^\dagger M.
$$

\subsection{Finite Sample Case}

Given $m$ observation $\Dcal_{X_1 X_2 X_3}=\{(x_1^i,x_2^i,x_3^i)\}_{i \in [m]}$ drawn~\iid~from a multi-view latent variable model $\PP(X_1,X_2,X_3)$, we now design a kernel algorithm to estimate the latent parameters from data. Although the empirical kernel embeddings can be infinite dimensional, we can carry out the decomposition using just the kernel matrices.
We will denote the implicitly feature matrix by
\begin{align*}
  \Phi &:= (\phi(x_1^1), \ldots, \phi(x_1^m), \phi(x_2^1),  \ldots, \phi(x_2^m)),  \\
  \Psi &:= (\phi(x_2^1), \ldots, \phi(x_2^m), \phi(x_1^1),  \ldots, \phi(x_1^m)),  
\end{align*}
and the corresponding kernel matrix by $K = \Phi^\top \Phi$ and $L = \Psi^\top \Psi$ respectively. 
Then the steps in the population case can be mapped one-by-one into kernel operations.   

{\bf Step 1.} We will perform a kernel eigenvalue decomposition of the empirical 2nd order embedding
$$
  \widehat \Ccal_{X_1 X_2}:= \frac{1}{2m} \sum_{i=1}^{m} \rbr{\phi(x_1^i) \otimes \phi(x_2^i) + \phi(x_2^i) \otimes \phi(x_1^i)},
$$
which can be expressed succinctly as $\widehat \Ccal_{X_1 X_2} = \frac{1}{2m} \Phi \Psi^\top$. 
Its leading $k$ eigenvectors $\widehat \Ucal_k = (\widehat u_1,\ldots,\widehat u_k)$ will lie in the span of the column of  $\Phi$,~\ie,~$\widehat \Ucal_k = \Phi (\beta_1,\ldots,\beta_k)$ with $\beta \in \RR^{2m}$. Then we can transform the eigen-value decomposition problem for an infinite dimensional matrix to a problem involving finite dimensional kernel matrices,
\begin{align*}
	\widehat \Ccal_{X_1 X_2}\, \widehat \Ccal_{X_1 X_2}^\top\, u = \widehat \sigma^2 \;u
	&~\Rightarrow~
	\frac{1}{4m^2}\Phi \Psi^\top \Psi \Phi^\top \Phi \beta = \widehat \sigma^2 \,\Phi \beta \\
	&~\Rightarrow~
	\frac{1}{4m^2} K L K \beta = \widehat \sigma^2 \,K \beta.
\end{align*}
Let the Cholesky decomposition of $K$ be $R^\top R$. Then by redefining $\widetilde{\beta}=R\beta$, and solving an eigenvalue problem
\begin{align}
 \frac{1}{4m^2} R L R^\top \widetilde{\beta} =\widehat  \sigma^2 \, \widetilde{\beta},~~\text{and obtain}~\beta = R^{\dagger} \widetilde{\beta}.
\end{align}
The resulting eigenvectors satisfy $u_i^\top u_{i'} = \beta_i^\top \Phi^\top \Phi \beta_{i'} =  \beta_{i}^\top K  \beta_{i'} =  \widetilde{\beta}_{i}^\top \widetilde{\beta}_{i'}=\delta_{ii'}$.
This step is summarized in Algorithm~\ref{alg:svd}.

\begin{algorithm}[t!]
\caption{KernelSVD($K$, $L$, $k$)}
% 	\textbf{In}: Two kernel matrices $K$ and $L$, and desired rank $r$ \\
	\textbf{Out}: $\widehat S_k$ and $(\beta_1,\ldots,\beta_k)$\\[-0.4cm]
  \begin{algorithmic}[1]
    \STATE Cholesky decomposition:\ $K=R^\top R$
    \STATE Eigen-decomposition:\ $\frac{1}{4m^2} R L R^\top \widetilde{\beta} = \widehat \sigma^2\,\widetilde{\beta}$
    \STATE Use $k$ leading eigenvalues:\ $\widehat S_k = \diag(\widehat \sigma_1,\ldots,\widehat \sigma_k)$
    \STATE Use $k$ leading eigenvectors:\ $(\widetilde{\beta}_1,\ldots,\widetilde{\beta}_k)$ to 
    compute:\ $(\beta_1,\ldots,\beta_k) = R^\dagger (\widetilde{\beta}_1,\ldots,\widetilde{\beta}_k)$
  \end{algorithmic}
  \label{alg:svd}
\end{algorithm}

{\bf Step 2.} We whiten the empirical 3rd order embedding
\begin{align*}
  &\widehat \Ccal_{X_1 X_2 X_3}:= \frac{1}{3m}\sum_{i=1}^{m} (\phi(x_1^i) \otimes \phi(x_2^i) \otimes \phi(x_3^i) \\
  &+ \phi(x_3^i) \otimes \phi(x_1^i) \otimes \phi(x_2^i) + \phi(x_2^i) \otimes \phi(x_3^i) \otimes \phi(x_1^i))
\end{align*}
using $\widehat \Wcal:= \widehat \Ucal_k \widehat S_k^{-1/2}$, and obtain 
\begin{align*}
  &\widehat \Tcal := \frac{1}{3m}\sum_{i=1}^m (\xi(x_1^i) \otimes \xi(x_2^i) \otimes \xi(x_3^i) \\
  &+ \xi(x_3^i) \otimes \xi(x_1^i) \otimes \xi(x_2^i) + \xi(x_2^i) \otimes \xi(x_3^i) \otimes \xi(x_1^i)),
\end{align*}
where
$$
	\xi(x_1^i) := \widehat S_k^{-1/2} (\beta_1,\ldots,\beta_k)^\top \Phi^\top \phi(x_1^i)~ \in~\RR^k.
$$

{\bf Step 3.} We run tensor power method~\cite{AnandkumarEtal:community12} on the finite dimension tensor $\widehat \Tcal$ to obtain its leading $k$ eigenvectors $\widehat M:=(\widehat v_1,\ldots,\widehat v_k)$.

{\bf Step 4.} The estimates of the conditional embeddings are
\begin{align*}
  \widehat \Ccal_{X|H} = (\widehat \mu_{X|h=1},\ldots, \widehat \mu_{X|h=k}) = \Phi (\beta_1,\ldots,\beta_k) \widehat  S_k^{1/2} \widehat M.
\end{align*}

\section{Sample Complexity}

Let $\kappa:=\sup_{x \in \Omega} k(x,x)$,   $\| \cdot\|_{}$ be the Hilbert-Schmidt norm, $\pi_{\min}:=\min_{i\in [k]} \pi_i$ and $\sigma_k(\Ccal_{X_1, X_2})$ be the $k$-th singular value of $\Ccal_{X_1, X_2}$.

\Note{can you provide a simplified version of the following theorem. and add some intuitive interpretation after that?}

\begin{theorem}[Sample Bounds]\label{thm:samplebound}
Pick  any $\delta\in (0,1)$. When the number of samples $m$ satisfies
\beq m >\frac{\rho\kappa^2  \log\frac{\delta}{2}}{\sigma_k(\Ccal_{X_1, X_2})}, 
\quad \rho:= \max\left(\frac{512  }{\sigma^3_k(\Ccal_{X_1, X_2})}, \frac{C' k^2 }{\sigma_k^{3}(\Ccal_{X_1, X_2})},\frac{C^{''} k^{\frac{2}{3}} }{\pi_{\min}^{\frac{1}{3}} }\right),\eeq for some constants $C', C^{''}>0$, and the number of iterations $N$  and  the number of random initialization vectors $L$  (drawn uniformly on the sphere $\mathcal{S}^{k-1}$)  satisfy
\begin{align*}
N \geq ...
% N \geq C_2 \cdot \biggl( \log(k) + \log\log\Bigl(
% \frac{1}{\sqrt{\pi}_{\min}\epsilon_T} \Bigr) \biggr),\\
% \sqrt{\frac{\ln(L/\log_2(k/\delta))}{\ln(k)}}
% \cdot \Biggl( 1 - \frac{\ln(\ln(L/\log_2(k/\delta))) +
% C_3}{4\ln(L/\log_2(k/\delta))} -
% \sqrt{\frac{\ln(8)}{\ln(L/\log_2(k/\delta))}} \Biggr)
% \geq 1.02 \Biggl( 1 + \sqrt{\frac{\ln(4)}{\ln(k)}}
% \Biggr)
% .
\end{align*}
for constants $C_2,C_3>0$. (Note that the condition on $L$ holds with $L = \poly(k) \log(1/\delta)$.) The robust power method in~\cite{AnandkumarEtal:community12} yields eigen-pairs $(\h{\lambda}_i, \h{\phi}_i)$ such that there exists a permutation $\eta$, with probability $1-4\delta$, we have
\begin{align*}
&\|\pi^{-1/2}_{j} \mu_{X|h=j}-\h{\phi}_{\eta(j)}\| \leq 8 \epsilon_T \cdot\pi^{-1/2}_{j}
, \\
&|\pi^{-1/2}_{j}-\h{\lambda}_{\eta(j)}| \leq  5\epsilon_T, \quad \forall j \in [k]
,
\end{align*}
and
\[
\biggl\|
T - \sum_{j=1}^k \hat\lambda_j \hat{\phi}_j^{\otimes 3}
\biggr\| \leq 55\eps_T,
\] where $\eps_T$ is the tensor perturbation bound
\[ \eps_T := \|\h{\Tcal} - \Tcal\| \leq 
\frac{12 \kappa \sqrt{\log\frac{\delta}{2}}}{\sqrt{m} \, \sigma_k^{1.5}(\Ccal_{X_1, X_2})} + \frac{512 \sqrt{2} \kappa^3 \left(\log\frac{\delta}{2}\right)^{1.5}}{m^{1.5} \,\sigma_k^{3}(\Ccal_{X_1, X_2}) \sqrt{\pi}_{\min}}\]

\end{theorem}

Thus, the above result provides bounds on the estimated eigen-pairs using the robust tensor power method.
The proof is in Appendix~\ref{app:samplebound}.

\section{Experiments}

{\bf Methods.} We compared our kernel nonparametric algorithm with three alternatives
\begin{enumerate}[noitemsep,nolistsep]
  \item The EM algorithm for mixture of Gaussians. The EM algorithm is not guaranteed to find the global solutions each trial. Thus we randomly initialize it $10$ times.  
  \item The spectral algorithm for mixture of spherical Gaussians~\citep{Hsu13}. The assumption in~\citet{Hsu13} is very restricted: the collection of spherical Gaussian centers need to span a $k$-dimension subpsace. 
  \item A discretization based spectral algorithm~\citep{Hiroyuki10}. This algorithm used histogram to approximate the joint distribution of the observed variables and then applied spectral algorithms to recover the discretized conditional density. It is well-known that density estimation using histogram is poor even for 3 dimensional data. The error of this algorithm is typically $10$ times larger than alternatives. To make the curves for other methods clearer, we did not plot the performance of \citet{Hiroyuki10} algorithm in the figures. 
\end{enumerate}
% It could be thought as a special case of our method using delta kernel. 
% It is predictable that their performance is not comparable to others because of the inferior histgram comparing to smooth kernel, 
Our method has a hyper-parameter, kernel bandwidth, which we selected for each view separately using held-out likelihoods. 

{\bf Data.} We generated three-dimensional synthetic data from various mixture models. The variables corresponding to the dimensions are independent given the latent component indicator. More specifically, we explore four settings
\begin{enumerate}[noitemsep,nolistsep]
  \item Single-view model with Gaussian conditional density
  \item Single-view model with Gamma conditional density
  \item Multi-view model with various Gaussian conditional densities
  \item Multi-view model with a mix of Gaussian and Gamma conditional densities
\end{enumerate}
We chose the skewness parameter of the Gamma distribution to reasonably large. Furthermore, we chose the mean and variance parameters of the Gaussian/Gamma density such that component pair-wise overlap is relatively small according to the fisher ratio $\frac{(\mu_1 - \mu_2)^2}{\sigma_1^2+\sigma_2^2}$.  

We also varied the number of samples $m$ for observed variables, from $50$ to $10,000$, and 
experimented with $k=2,3,4$ or $8$ mixture components. The mixture proportion for the $h$-th component is set to be $\pi_h = \frac{2i}{k(k+1)},~\forall h\in[k]$ (unbalanced). It is worth noting that as $k$ becomes larger, it is more difficult to recover parameters. This is because only a small number of data will be generated for the first several clusters. For each $n,k$ with each setting, we randomly generate 10 sets of samples and report the average results. 

{\bf Error measure.} We measured the performance of algorithms by the 
following weighted $\ell_2$ norm difference
\begin{align*}
  Err:=\sum_{h=1}^{k} \pi_h \, \sqrt{\sum_{i=1}^{m'} (p(x^i|h) - \widehat{p}(x^i|h))^2 }
\end{align*}

{\bf Result.} The results are plotted in Figure~\ref{}. It is clear that both the nonparametric method and EM converge rapidly with the data increment in all experiments setting. 

In mixture of Gaussian setting, the EM algorithm is best because the setting is fully satisfied to its assumption. While the spectral learning algorithm for sphereical Gaussian didn't perform well since the assumption for the method is too restrict. It is reasonable that our nonparametric is a little bit worse than EM algorithm because of the complexity nonparametric form of probability distribution.

In the mixture of Gaussian/Gamma setting, the EM algorithm is good when data is less. However, our nonparametric spectral algorithm provided the best results when sample size is large enough. This phenomena demonstrated the trade-off between model complexity and sample size. 

\Note{result description needs refinement.} 

\bibliographystyle{icml2014}
\bibliography{../nonparametric_mixture,../bibfile,../experiment/mlv_kernel}

\section{Sample Complexity Analysis}

\subsection{Robust Tensor Power Method}
We recap the robust tensor power method for finding the tensor eigen-pairs, analyzed in detail in~\cite{AnandkumarEtal:community12}.

\begin{algorithm}
\caption{$\{\lambda, \Phi\}\leftarrow $TensorEigen$(T,\, \{v_i\}_{i\in [L]}, N)$}\label{alg:robustpower}
\begin{algorithmic}
\renewcommand{\algorithmicrequire}{\textbf{Input: }}
\renewcommand{\algorithmicensure}{\textbf{Output: }}
\REQUIRE Tensor $T\in \R^{k \times k \times k}$, set of $L$ initialization vectors $\{v_i\}_{i\in L}$, number of
iterations  $N$.
%\ENSURE Eigenpairs:  $\lambda$ is the vector of eigenvalues and $\Phi$ is the matrix of eigenvectors of  $T$.
\ENSURE the estimated eigenvalue/eigenvector pairs $\{\lambda, \Phi\}$, where $\lambda$ is the vector of eigenvalues and $\Phi$ is the matrix of eigenvectors.

\FOR{$i =1$ to $k$}
\FOR{$\tau = 1$ to $L$}
\STATE $\th{0}\leftarrow v_\tau$.
\FOR{$t = 1$ to $N$}
\STATE $\tilde{T}\leftarrow T$.
\FOR{$j=1$ to $i-1$ (when $i>1$)}
\IF{$|\lambda_j \inner{\th{t}^{(\tau)}, \phi_j}|>\xi$}
\STATE $\tilde{T}\leftarrow \tilde{T}- \lambda_j \phi_j^{\otimes 3}$.
\ENDIF
\ENDFOR

\STATE Compute power iteration update
$
\th{t}^{(\tau)}  :=
\frac{\tilde{T}(I, \th{t-1}^{(\tau)}, \th{t-1}^{(\tau)})}
{\|\tilde{T}(I, \th{t-1}^{(\tau)}, \th{t-1}^{(\tau)})\|}
$\ENDFOR
\ENDFOR

\STATE Let $\tau^* := \arg\max_{\tau \in L} \{ \tilde{T}(\th{N}^{(\tau)},
\th{N}^{(\tau)}, \th{N}^{(\tau)}) \}$.

\STATE Do $N$ power iteration updates starting from
$\th{N}^{(\tau^*)}$ to obtain eigenvector estimate $\phi_i$, and set $\lambda_i :=
\tilde{T}(\phi_i, \phi_i, \phi_i)$.

\ENDFOR
\RETURN the estimated eigenvalue/eigenvectors
$(\lambda, \Phi)$.

\end{algorithmic}
\end{algorithm}


We now recap the result of~\cite[Thm. 13]{AnandkumarEtal:community12} that establishes bounds on the eigen-estimates under good initialization vectors for the above procedure. 
Let $\Tcal=\sum_{i\in [k]}\lambda_i v_i$, where $v_i$ are orthonormal vectors and $\lambda_1\geq \lambda_2\geq\ldots \lambda_k$. Let $\h{\Tcal}=\Tcal+E$ be the perturbed tensor with $\|E\|\leq \epsilon_{T}$. Recall that $N$ denotes the number of iterations of the tensor power method.
We call an initialization vector $u$ to be $(\gamma, R_0)$-good  if there exists $v_i$ such that $\inner{u, v_i}> R_0$
  and $|\inner{u, v_i}| -\max_{j<i} |\inner{u,v_j}| > \gamma  |\inner{u,v_i}|$.   Choose $\gamma=1/100$.


\begin{theorem}
\label{thm:robustpower}
There exists universal constants $C_1, C_2 > 0$  such that the
following holds.
\beq\label{eqn:robustpowerconditions}
\epsilon_{T} \leq C_1 \cdot \lambda_{\min} R_0^2,
\qquad
N \geq C_2 \cdot \left( \log(k) + \log\log\left(
\frac{\lambdamax}{\epsilon_{tensor}} \right) \right)
,
\eeq Assume there is at least one good initialization vector corresponding to each $v_i$, $i\in [k]$. The parameter $\xi$ for choosing deflation vectors in each iteration of the tensor power method in Procedure~\ref{alg:robustpower}  is chosen as $\xi\geq 25 \eps$. We obtain  eigenvalue-eigenvector pairs  $(\hat\lambda_1,\hat{v}_1), (\hat\lambda_2,\hat{v}_2), \dotsc,
(\hat\lambda_k,\hat{v}_k)$ such that  there exists a permutation $\pi$ on
$[k]$ with
\[
\|v_{\pi(j)}-\hat{v}_j\| \leq 8 \epsilon/\lambda_{\pi(j)}
, \qquad
|\lambda_{\pi(j)}-\hat\lambda_j| \leq 5\epsilon , \quad \forall j \in [k]
,
\]
and
\[
\left\|
\Tcal - \sum_{j=1}^k \hat\lambda_j \hat{v}_j^{\otimes 3}
\right\| \leq 55\eps .
\]
\end{theorem}

In the sequel, we establish concentration bounds that allows us to translate the above condition on tensor perturbation~\eqref{eqn:robustpowerconditions}  to sample complexity bounds.

\subsection{Concentration Bounds}

\subsubsection{Analysis of Whitening} 

Recall that we use the covariance operator $\Ccal_{X_1 X_2}$ for whitening the 3rd order embedding $\Ccal_{X_1, X_2, X_3}$. We first analyze the perturbation in whitening when sample estimates are employed. 

Let $\h{\Ccal}_{X_1 X_2}$ denote the sample covariance operator between variables $X_1$ and $X_2$, and let \[B:=0.5(\h{\Ccal}_{X_1 X_2}+ \h{\Ccal}_{X_1 X_2}^\top)=\h{\Ucal}\h{S}\h{\Ucal}^\top\] denote the SVD.
Let $\Ucal_k$ and $S_k$ denote the restriction to top-$k$ eigen-pairs, and let $B_{k} := \Ucal_k S_k \Ucal_k^\top$. Recall that the whitening matrix is given by $\h{\Wcal}:=\h{\Ucal}_k \h{S}_k^{-1/2}$. Now $\h{\Wcal}$ whitens $B_k$, i.e. $\h{\Wcal}^\top B_{k} \h{\Wcal}=I$.

Now consider the SVD of
\[ \h{\Wcal}^\top \Ccal_{X_1 X_2} \h{\Wcal}= A D A^\top,\] and define \[\Wcal:= \h{\Wcal} AD^{-1/2}A^\top, \] and $\Wcal$ whitens $\Ccal_{X_1 X_2}$ since $\Wcal^\top  \Ccal_{X_1 X_2} W=I$.
Recall that by exchangeability assumption, 
\[ \Ccal_{X_1,X_{2}}
  = \sum_{j=1}^k \pi_j \cdot \mu_{X|j} \otimes \mu_{X|j} \tcr{+  E_{X_1 X_2}}= M \Diag(\pi) M^\top  \tcr{+E_{X_1 X_2}}, \] where the $j^{\tha}$ column of $M$, $M_j = \mu_{X|j}$.

We now establish the following perturbation bound on the whitening procedure. Recall from \eqref{eqn:deltapairs}, $ \epsilon_{pairs}:=\nbr{\Ccal_{X_1,X_{2}} - \widehat \Ccal_{X_1,X_{2}}}_{}$. Let $\sigma_1(\cdot) \geq \sigma_2(\cdot)\ldots$ denote the singular values of an operator.

\begin{lemma}[Whitening perturbation]\label{lemma:whiten} Assuming that $\epsilon_{pairs} < 0.5 \sigma_k(\Ccal_{X_1 X_2})$,
\beq \epsilon_{W}:= \|\Diag(\pi)^{1/2}M^\top(\h{\Wcal}-\Wcal)\|\leq \frac{2(2\epsilon_{pairs} \tcr{+\sigma_{k+1}(\Ccal_{X_1 X_2})})}{ \sigma_{k}(\Ccal_{X_1 X_2})}\tcr{\cdot (1+ \sigma_{k+1}(\Ccal_{X_1 X_2}))}\eeq
\end{lemma}

\paragraph{Remark: }Note that $\sigma_{k}(\Ccal_{X_1 X_2}) = \sigma_{k}^2(M)$.

\bprf The proof is along the lines of~\cite[Lemma 16]{AnandkumarEtal:community12}, but adapted to whitening using the covariance operator here.
 \begin{align*}\|\Diag(\pi)^{1/2} M^\top(\h{\Wcal}-\Wcal)\|&= 
\|\Diag(\pi)^{1/2} M^\top W(A D^{1/2} A^\top -I)\|\\ &\leq\|\Diag(\pi)^{1/2} M^\top \Wcal\| \|D^{1/2}-I\|. \end{align*} Since $\Wcal$ whitens $\Ccal_{X_1 X_2}=M \Diag(\pi) M^\top\tcr{+E}$, we have that $\|\Diag(\pi)^{1/2} M^\top \Wcal\|=1$ \tcr{or $\|\Diag(\pi)^{1/2} M^\top \Wcal\|\leq \|I-E\|^{1/2}\leq 1+ \sigma_{k+1}(\Ccal_{X_1 X_2})$}. Now we control $\|D^{1/2}-I\|$.  Let $\tl{E}:= \Ccal_{X_1,X_{2}} -B_k$, where recall that $B=0.5( \widehat \Ccal_{X_1,X_{2}}+ \h{\Ccal}_{X_1 X_2}^\top)$ and $B_k$ is its restriction to top-$k$ singular values. Thus, we have $\|\tl{E}\|_{HS} \leq \epsilon_{pairs} + \sigma_{k+1}(B)\leq 2\epsilon_{pairs}\tcr{+\sigma_{k+1}(\Ccal_{X_1 X_2})} $.
 We now have
\begin{align*}
\|D^{1/2}-I\|&\leq \|(D^{1/2}-I)(D^{1/2}+I)\|\leq \|D-I\|
\\ &=\|AD A^\top - I\| = \|\h{\Wcal}^\top \Ccal_{X_1 X_2}  \h{\Wcal} -I\|\\ &=\| \h{\Wcal}^\top  \tl{E} \h{\Wcal}\| \leq \|\h{\Wcal}\|^2 ( 2 \epsilon_{pairs}\tcr{+\sigma_{k+1}(\Ccal_{X_1 X_2})}).
\end{align*}Now
\[ \|\h{\Wcal}^2\| \leq\frac{1}{ \sigma_k(\h{\Ccal}_{X_1 X_2})}\leq \frac{2}{\sigma_k(\Ccal_{X_1 X_2})},\] when  $\epsilon_{pairs}<0.5 \sigma_k(\Ccal_{X_1 X_2})$.
\eprf

\subsubsection{Tensor Concentration Bounds}

Recall that the whitened tensor from samples is given by
$$\h{\Tcal} := \h{\Ccal}_{X_1 X_2 X_3} \times_1 (\h{\Wcal}^\top) \times_2 (\h{\Wcal}^\top) \times_3 (\h{\Wcal}^\top).$$ We want to establish its perturbation from the whitened tensor using exact statistics
$$\Tcal := \Ccal_{X_1 X_2 X_3} \times_1 (\Wcal^\top ) \times_2 (\Wcal^\top ) \times_3 (\Wcal^\top ).$$ Further, we have 
$$\Ccal_{X_1 X_2 X_3}= \sum_{h\in [k]} \pi_h \cdot \mu_{X|h} \otimes \mu_{X|h} \otimes \mu_{X|h} \tcr{+ E_{X_1, X_2, X_3}}$$

Let $\epsilon_{triples}:= \|\h{\Ccal}_{X_1 X_2 X_3}-\Ccal_{X_1 X_2 X_3}\|_{}$. Let $\pi_{\min}:=\min_{h\in [k]}\pi_h$.

\begin{lemma}[Tensor perturbation bound]
Assuming that $\epsilon_{pairs} < 0.5 \sigma_k(\Ccal_{X_1 X_2})$, we have
\beq\label{eqn:epsilonT} \epsilon_T:= \|\h{\Tcal} - \Tcal\|
\leq \frac{2\sqrt{2}\epsilon_{triples}}{\sigma_k(\Ccal_{X_1 X_2})^{1.5} }+\frac{\epsilon_W^3}{\sqrt{\pi_{\min}}}
\tcr{+ \|E_{X_1 X_2 X_3}\| \frac{\epsilon_W^3}{\pi_{\min}^{1.5} \sigma_k(M)^3} }.\eeq
\end{lemma} 


\bprf  Define
  intermediate tensor
\begin{align*} \tl{\Tcal}&:= \Ccal_{X_1 X_2 X_3} \times_1 (\h{\Wcal}^\top) \times_2 (\h{\Wcal}^\top) \times_3 (\h{\Wcal}^\top).\end{align*}
We will bound $\|\h{\Tcal}-\tl{\Tcal}\|$  and $\| \h{\Tcal}-\Tcal\|$  separately. 
\begin{align*}
\|\h{\Tcal}-\tl{\Tcal}\| &\leq \|\h{\Ccal}_{X_1, X_2, X_2} - \Ccal_{X_1, X_2, X_3}\| \|\h{\Wcal}\|^3\leq \frac{2\sqrt{2}\epsilon_{triples}}{\sigma_k(\Ccal_{X_1 X_2})^{1.5} },
\end{align*}using the bound on $\|\h{\Wcal}\|$ in Lemma~\ref{lemma:whiten}. For the other term,
first note that
\[ \Ccal_{X_1, X_2, X_3} = \sum_{h\in [k]} \pi_h \cdot M_h \otimes M_h \otimes M_h \tcr{+ E_{X_1, X_2, X_3}}, \]
 \tcr{ where $\|E_{X_1, X_2, X_3}\|$ is the residual and we need to bound this in non-parametric case.}
\begin{align*} \|\h{\Tcal}-\Tcal\|&= \| \Ccal_{X_1 X_2 X_3}\times_1 (\h{\Wcal} -\Wcal)^\top \times_2 (\h{\Wcal} -\Wcal)^\top \times_3 (\h{\Wcal}-\Wcal)^\top\| \\
&\leq \frac{ \| \Diag(\pi)^{1/2}M^\top(\h{\Wcal}-\Wcal)\|^3}{\sqrt{\pi_{\min}}}
\tcr{+ \|E_{X_1 X_2 X_3}\| \|\h{\Wcal}-\Wcal\|^3}\\
&= \frac{\epsilon_W^3}{\sqrt{\pi_{\min}}}
\tcr{+ \|E_{X_1 X_2 X_3}\| \frac{\epsilon_W^3}{\pi_{\min}^{1.5} \sigma_k(M)^3} }
\end{align*}
\eprf\\


We obtain a condition on the above perturbation $\epsilon_T$ in \eqref{eqn:epsilonT} by applying Theorem~\ref{thm:robustpower} as
$ \epsilon_T\leq C_1\lambda_{\min} R_0^2$. Here, we have $\lambda_{i} = 1/\sqrt{\pi_{i}}\geq 1$. For random initialization, we have that $R_0 \sim 1/\sqrt{k}$, with probability $1-\delta$ using $\poly(k) \poly(1/\delta)$ trials~\cite[Thm. 5.1]{AnandkumarEtal:tensor12}. Thus, we require that $ \epsilon_T  \leq \frac{C_1}{k}$. Summarizing, we require for the following conditions to hold 
\beq\epsilon_{pairs}\leq 0.5 \sigma_k(\Ccal_{X_1 X_2}), \quad \epsilon_T  \leq \frac{C_1}{k}.\eeq

\aacomment{need to substitute for $\epsilon_{pairs}$ and $\epsilon_{triples}$ and if we bound the residuals, we are done.}

\subsubsection{Concentration bounds for Empirical Operators}

Concentration results for the singular value decomposition of empirical operators.

\begin{lemma} Let $\kappa:=\sup_{x \in \Omega} k(x,x)$, and $\| \cdot\|_{}$ be the Hilbert-Schmidt norm, we have for \beq \epsilon_{pairs}:=\nbr{\Ccal_{X_1 X_2} - \widehat \Ccal_{X_1 X_2}}_{},\label{eqn:deltapairs} \eeq
\begin{eqnarray}
	\Pr \cbr{\epsilon_{pairs}  \leqslant \frac{2\sqrt{2}\kappa \sqrt{\delta}}{\sqrt{m}} } \geqslant 1-2\exp(-\delta). \label{eq:operator_concentration}
\end{eqnarray}
\end{lemma}

\begin{proof}
We will use similar arguments as in~\cite{RosBelVit2010} which deals with symmetric operator. Let $\xi_{i}$ be defined as
\begin{eqnarray}
\xi_{i}\, =\, \phi(x_t^i) \otimes \phi(x_{t'}^i) - \Ccal_{X_t,X_{t'}}.
\end{eqnarray}
It is easy to see that $\mathbb{E}[\xi_{i}] = 0$. Further, we have
\begin{eqnarray}
	\sup_{x_1,x_2} \nbr{\phi(x_1) \otimes \phi(x_2)}^{2}_{} \leqslant \kappa^{2},
\end{eqnarray}
which implies that $\nbr{\Ccal_{X_1 X_2}}_{} \leqslant \kappa$, and $\nbr{\xi_i}_{} \leqslant 2 \kappa$. The result then follows from the Hoeffding's inequality in Hilbert space.
\end{proof}

Similarly, we have the concentration bound for 3rd order embedding

\begin{lemma} Let $\kappa:=\sup_{x \in \Omega} k(x,x)$, and $\| \cdot\|_{}$ be the Hilbert-Schmidt norm, we have for \beq \epsilon_{pairs}:=\nbr{\Ccal_{X_1 X_2 X_3} - \widehat \Ccal_{X_1 X_2 X_3}}_{},\label{eqn:deltapairs} \eeq
\begin{eqnarray}
	\Pr \cbr{\epsilon_{triples}  \leqslant \frac{3\sqrt{2}\kappa \sqrt{\delta}}{\sqrt{m}} } \geqslant 1-2\exp(-\delta). \label{eq:operator_concentration2}
\end{eqnarray}
\end{lemma}


\end{document}