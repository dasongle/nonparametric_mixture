\section{Sample Complexity Analysis}

\subsection{Robust Tensor Power Method}
We recap the robust tensor power method for finding the tensor eigen-pairs, analyzed in detail in~\cite{AnandkumarEtal:community12}.

\begin{algorithm}
\caption{$\{\lambda, \Phi\}\leftarrow $TensorEigen$(T,\, \{v_i\}_{i\in [L]}, N)$}\label{alg:robustpower}
\begin{algorithmic}
\renewcommand{\algorithmicrequire}{\textbf{Input: }}
\renewcommand{\algorithmicensure}{\textbf{Output: }}
\REQUIRE Tensor $T\in \R^{k \times k \times k}$, set of $L$ initialization vectors $\{v_i\}_{i\in L}$, number of
iterations  $N$.
%\ENSURE Eigenpairs:  $\lambda$ is the vector of eigenvalues and $\Phi$ is the matrix of eigenvectors of  $T$.
\ENSURE the estimated eigenvalue/eigenvector pairs $\{\lambda, \Phi\}$, where $\lambda$ is the vector of eigenvalues and $\Phi$ is the matrix of eigenvectors.

\FOR{$i =1$ to $k$}
\FOR{$\tau = 1$ to $L$}
\STATE $\th{0}\leftarrow v_\tau$.
\FOR{$t = 1$ to $N$}
\STATE $\tilde{T}\leftarrow T$.
\FOR{$j=1$ to $i-1$ (when $i>1$)}
\IF{$|\lambda_j \inner{\th{t}^{(\tau)}, \phi_j}|>\xi$}
\STATE $\tilde{T}\leftarrow \tilde{T}- \lambda_j \phi_j^{\otimes 3}$.
\ENDIF
\ENDFOR

\STATE Compute power iteration update
$
\th{t}^{(\tau)}  :=
\frac{\tilde{T}(I, \th{t-1}^{(\tau)}, \th{t-1}^{(\tau)})}
{\|\tilde{T}(I, \th{t-1}^{(\tau)}, \th{t-1}^{(\tau)})\|}
$\ENDFOR
\ENDFOR

\STATE Let $\tau^* := \arg\max_{\tau \in L} \{ \tilde{T}(\th{N}^{(\tau)},
\th{N}^{(\tau)}, \th{N}^{(\tau)}) \}$.

\STATE Do $N$ power iteration updates starting from
$\th{N}^{(\tau^*)}$ to obtain eigenvector estimate $\phi_i$, and set $\lambda_i :=
\tilde{T}(\phi_i, \phi_i, \phi_i)$.

\ENDFOR
\RETURN the estimated eigenvalue/eigenvectors
$(\lambda, \Phi)$.

\end{algorithmic}
\end{algorithm}


We now recap the result of~\cite[Thm. 13]{AnandkumarEtal:community12} that establishes bounds on the eigen-estimates under good initialization vectors for the above procedure. 
Let $\Tcal=\sum_{i\in [k]}\lambda_i v_i$, where $v_i$ are orthonormal vectors and $\lambda_1\geq \lambda_2\geq\ldots \lambda_k$. Let $\h{\Tcal}=\Tcal+E$ be the perturbed tensor with $\|E\|\leq \epsilon_{T}$. Recall that $N$ denotes the number of iterations of the tensor power method.
We call an initialization vector $u$ to be $(\gamma, R_0)$-good  if there exists $v_i$ such that $\inner{u, v_i}> R_0$
  and $|\inner{u, v_i}| -\max_{j<i} |\inner{u,v_j}| > \gamma  |\inner{u,v_i}|$.   Choose $\gamma=1/100$.


\begin{theorem}
\label{thm:robustpower}
There exists universal constants $C_1, C_2 > 0$  such that the
following holds.
\beq\label{eqn:robustpowerconditions}
\epsilon_{T} \leq C_1 \cdot \lambda_{\min} R_0^2,
\qquad
N \geq C_2 \cdot \left( \log(k) + \log\log\left(
\frac{\lambdamax}{\epsilon_{tensor}} \right) \right)
,
\eeq Assume there is at least one good initialization vector corresponding to each $v_i$, $i\in [k]$. The parameter $\xi$ for choosing deflation vectors in each iteration of the tensor power method in Procedure~\ref{alg:robustpower}  is chosen as $\xi\geq 25 \eps$. We obtain  eigenvalue-eigenvector pairs  $(\hat\lambda_1,\hat{v}_1), (\hat\lambda_2,\hat{v}_2), \dotsc,
(\hat\lambda_k,\hat{v}_k)$ such that  there exists a permutation $\pi$ on
$[k]$ with
\[
\|v_{\pi(j)}-\hat{v}_j\| \leq 8 \epsilon/\lambda_{\pi(j)}
, \qquad
|\lambda_{\pi(j)}-\hat\lambda_j| \leq 5\epsilon , \quad \forall j \in [k]
,
\]
and
\[
\left\|
\Tcal - \sum_{j=1}^k \hat\lambda_j \hat{v}_j^{\otimes 3}
\right\| \leq 55\eps .
\]
\end{theorem}

In the sequel, we establish concentration bounds that allows us to translate the above condition on tensor perturbation~\eqref{eqn:robustpowerconditions}  to sample complexity bounds.

\subsection{Concentration Bounds}

\subsubsection{Analysis of Whitening} 

Recall that we use the covariance operator $\Ccal_{X_1,X_2}$ for whitening the third-order operator \aacomment{a better term for this?} $\Ccal_{X_1, X_2, X_3}$. We first analyze the perturbation in whitening when sample estimates are employed. 

Let $\h{\Ccal}_{X_1,X_2}$ denote the sample covariance operator between variables $X_1$ and $X_2$, and let \[B:=0.5(\h{\Ccal}_{X_1,X_2}+ \h{\Ccal}_{X_1,X_2}^\top)=\h{U}\h{S}\h{U}^\top\] denote the SVD.
Let $U_k$ and $S_k$ denote the restriction to top-$k$ eigen-pairs, and let $B_{k} := U_k S_k U_k^\top$. Recall that the whitening matrix is given by $\h{W}:=\h{U}_k \h{S}_k^{-1/2}$. Now $\h{W}$ whitens $B_k$, i.e. $\h{W}^\top B_{k} \h{W}=I$.

Now consider the SVD of
\[ \h{W}^\top \Ccal_{X_1,X_2} \h{W}= A D A^\top,\] and define \[W:= \h{W} AD^{-1/2}A^\top, \] and $W$ whitens $\Ccal_{X_1,X_2}$ since $W^\top \Ccal_{X_1, X_2} W=I$.
Recall that by exchangeability assumption, 
\[ \Ccal_{X_1,X_{2}}
  = \sum_{j=1}^k \pi_j \cdot \mu_{X|j} \otimes \mu_{X|j} \tcr{+  E_{X_1, X_2}}= M \Diag(\pi) M^\top  \tcr{+E_{X_1, X_2}}, \] where the $j^{\tha}$ column of $M$, $M_j = \mu_{X|j}$.

We now establish the following perturbation bound on the whitening procedure. Recall from \eqref{eqn:deltapairs}, $ \epsilon_{pairs}:=\nbr{\Ccal_{X_1,X_{2}} - \widehat \Ccal_{X_1,X_{2}}}_{HS}$. Let $\sigma_1(\cdot) \geq \sigma_2(\cdot)\ldots$ denote the singular values of a matrix.\aacomment{will have to define the generalization to infinite dimensional objects}

\aacomment{Spectral norm will have to be replaced by HS norm. My suggestion is to use $\| \cdot\|$ to denote HS norm.}

\begin{lemma}[Whitening perturbation]\label{lemma:whiten} Assuming that $\epsilon_{pairs} < 0.5 \sigma_k(\Ccal_{X_1, X_2})$,
\beq \epsilon_{W}:= \|\Diag(\pi)^{1/2}M^\top(\h{W}-W)\|\leq \frac{2(2\epsilon_{pairs} \tcr{+\sigma_{k+1}(\Ccal_{X_1,X_2})})}{ \sigma_{k}(\Ccal_{X_1, X_2})}\tcr{\cdot (1+ \sigma_{k+1}(\Ccal_{X_1, X_2}))}\eeq
\end{lemma}

\paragraph{Remark: }Note that $\sigma_{k}(\Ccal_{X_1, X_2}) = \sigma_{k}^2(M)$.

\aacomment{term in red for the more general case}

\bprf The proof is along the lines of~\cite[Lemma 16]{AnandkumarEtal:community12}, but adapted to whitening using the covariance operator here.
 \begin{align*}\|\Diag(\pi)^{1/2} M^\top(\h{W}-W)\|&= 
\|\Diag(\pi)^{1/2} M^\top W(A D^{1/2} A^\top -I)\|\\ &\leq\|\Diag(\pi)^{1/2} M^\top W\| \|D^{1/2}-I\|. \end{align*} Since $W$ whitens $\Ccal_{X_1, X_2}=M \Diag(\pi) M^\top\tcr{+E}$, we have that $\|\Diag(\pi)^{1/2} M^\top W\|=1$ \tcr{or $\|\Diag(\pi)^{1/2} M^\top W\|\leq \|I-E\|^{1/2}\leq 1+ \sigma_{k+1}(\Ccal_{X_1, X_2})$}. Now we control $\|D^{1/2}-I\|$.  Let $\tl{E}:= \Ccal_{X_1,X_{2}} -B_k$, where recall that $B=0.5( \widehat \Ccal_{X_1,X_{2}}+ \h{\Ccal}_{X_1, X_2}^\top)$ and $B_k$ is its restriction to top-$k$ singular values. Thus, we have $\|\tl{E}\|_{HS} \leq \epsilon_{pairs} + \sigma_{k+1}(B)\leq 2\epsilon_{pairs}\tcr{+\sigma_{k+1}(\Ccal_{X_1,X_2})} $.
 We now have
\begin{align*}
\|D^{1/2}-I\|&\leq \|(D^{1/2}-I)(D^{1/2}+I)\|\leq \|D-I\|
\\ &=\|AD A^\top - I\| = \|\h{W}^\top \Ccal_{X_1, X_2}  \h{W} -I\|\\ &=\| \h{W}^\top  \tl{E} \h{W}\| \leq \|\h{W}\|^2 ( 2 \epsilon_{pairs}\tcr{+\sigma_{k+1}(\Ccal_{X_1,X_2})}).
\end{align*}Now
\[ \|\h{W}^2\| \leq\frac{1}{ \sigma_k(\h{\Ccal}_{X_1, X_2})}\leq \frac{2}{\sigma_k(\Ccal_{X_1, X_2})},\] when  $\epsilon_{pairs}<0.5 \sigma_k(\Ccal_{X_1, X_2})$.
\eprf

\subsubsection{Tensor Concentration Bounds}

Recall that the whitened tensor from samples is given by
$$\h{\Tcal} := \h{\Ccal}_{X_1,X_2,X_3} \times_1 (\h{W}^\top) \times_2 (\h{W}^\top) \times_3 (\h{W}^\top).$$ We want to establish its perturbation from the whitened tensor using exact statistics
$$\Tcal := \Ccal_{X_1,X_2,X_3} \times_1 (W^\top) \times_2 (W^\top) \times_3 (W^\top).$$ Further, we have 
$$\Ccal_{X_1,X_2,X_3}= \sum_{i\in [k]} \pi_i \mu_{X|i}^{\otimes 3}\tcr{+ E_{X_1, X_2, X_3}}$$

Let $\epsilon_{triples}:= \|\h{\Ccal}_{X_1,X_2,X_3}-\Ccal_{X_1,X_2,X_3}\|_{HS}$. Let $\pi_{\min}:=\min_{i\in [k]}\pi_i$.

\begin{lemma}[Tensor perturbation bound]
Assuming that $\epsilon_{pairs} < 0.5 \sigma_k(\Ccal_{X_1, X_2})$, we have
\beq\label{eqn:epsilonT} \epsilon_T:= \|\h{\Tcal} - \Tcal\|
\leq \frac{2\sqrt{2}\epsilon_{triples}}{\sigma_k(\Ccal_{X_1, X_2})^{1.5} }+\frac{\epsilon_W^3}{\sqrt{\pi_{\min}}}
\tcr{+ \|E_{X_1,X_2,X_3}\| \frac{\epsilon_W^3}{\pi_{\min}^{1.5} \sigma_k(M)^3} }.\eeq
\end{lemma} 


\bprf  Define
  intermediate tensor
\begin{align*} \tl{\Tcal}&:= \Ccal_{X_1,X_2,X_3} \times_1 (\h{W}^\top) \times_2 (\h{W}^\top) \times_3 (\h{W}^\top).\end{align*}
We will bound $\|\h{\Tcal}-\tl{\Tcal}\|$  and $\| \h{\Tcal}-\Tcal\|$  separately. 
\begin{align*}
\|\h{\Tcal}-\tl{\Tcal}\| &\leq \|\h{\Ccal}_{X_1, X_2, X_2} - \Ccal_{X_1, X_2, X_3}\| \|\h{W}\|^3\leq \frac{2\sqrt{2}\epsilon_{triples}}{\sigma_k(\Ccal_{X_1, X_2})^{1.5} },
\end{align*}using the bound on $\|\h{W}\|$ in Lemma~\ref{lemma:whiten}. For the other term,
first note that
\[ \Ccal_{X_1, X_2, X_3} = \sum_{i\in [k]} \pi_i M_i^{\otimes 3}\tcr{+ E_{X_1, X_2, X_3}}, \]
 \tcr{ where $\|E_{X_1, X_2, X_3}\|$ is the residual and we need to bound this in non-parametric case.}
\begin{align*} \|\h{\Tcal}-\Tcal\|&= \| \Ccal_{X_1, X_2, X_3}\times_1 (\h{W} -W)^\top \times_2 (\h{W} -W)^\top \times_3 (\h{W}-W)^\top\| \\
&\leq \frac{ \| \Diag(\pi)^{1/2}M^\top(\h{W}-W)\|^3}{\sqrt{\pi_{\min}}}
\tcr{+ \|E_{X_1,X_2,X_3}\| \|\h{W}-W\|^3}\\
&= \frac{\epsilon_W^3}{\sqrt{\pi_{\min}}}
\tcr{+ \|E_{X_1,X_2,X_3}\| \frac{\epsilon_W^3}{\pi_{\min}^{1.5} \sigma_k(M)^3} }
\end{align*}
\eprf\\


We obtain a condition on the above perturbation $\epsilon_T$ in \eqref{eqn:epsilonT} by applying Theorem~\ref{thm:robustpower} as
$ \epsilon_T\leq C_1\lambda_{\min} R_0^2$. Here, we have $\lambda_{i} = 1/\sqrt{\pi_{i}}\geq 1$. For random initialization, we have that $R_0 \sim 1/\sqrt{k}$, with probability $1-\delta$ using $\poly(k) \poly(1/\delta)$ trials~\cite[Thm. 5.1]{AnandkumarEtal:tensor12}. Thus, we require that $ \epsilon_T  \leq \frac{C_1}{k}$. Summarizing, we require for the following conditions to hold 
\beq\epsilon_{pairs}\leq 0.5 \sigma_k(\Ccal_{X_1, X_2}), \quad \epsilon_T  \leq \frac{C_1}{k}.\eeq

\aacomment{need to substitute for $\epsilon_{pairs}$ and $\epsilon_{triples}$ and if we bound the residuals, we are done.}

\subsubsection{Concentration bounds for Empirical Operators}

Concentration results for the singular value decomposition of empirical operators.

\begin{theorem} Let $\kappa:=\sup_{x \in \Omega} k(x,x)$, and $\| \cdot\|_{HS}$ be the Hilbert-Schmidt norm, we have for \beq \epsilon_{pairs}:=\nbr{\Ccal_{X_t,X_{t'}} - \widehat \Ccal_{X_t,X_{t'}}}_{HS},\label{eqn:deltapairs} \eeq
\begin{eqnarray}
	\PP \cbr{\epsilon_{pairs}  \leqslant \frac{2\sqrt{2}\kappa \sqrt{\delta}}{\sqrt{n}} } \geqslant 1-2\exp(-\delta). \label{eq:operator_concentration}
\end{eqnarray}
\end{theorem}

\begin{proof}
We will use similar arguments as in~\cite{RosBelVit2010} which deals with symmetric operator. Let $\xi_{i}$ be defined as
\begin{eqnarray}
\xi_{i}\, =\, \phi(x_t^i) \otimes \phi(x_{t'}^i) - \Ccal_{X_t,X_{t'}}.
\end{eqnarray}
It is easy to see that $\mathbb{E}[\xi_{i}] = 0$. Further, we have
\begin{eqnarray}
	\sup_{x_t,x_{t'}} \nbr{\phi(x_t) \otimes \phi(x_{t'})}^{2}_{HS} \leqslant \kappa^{2},
\end{eqnarray}
which implies that $\nbr{\Ccal_{X_t,X_{t'}}}_{HS} \leqslant \kappa$, and $\nbr{\xi_i}_{HS} \leqslant 2 \kappa$. The result then follows from the Hoeffding's inequality in Hilbert space.
\end{proof}


Use analysis for tensor paper and then concentration inequality for Hilbert space embeddings.
 