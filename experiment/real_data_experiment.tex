%%%%%%%%%%%%%%%%%%%%%%%%%%%%%%%%%%%%%%%%%%%%%%%%%%%%%%%%%%%%%%%%%%
%%%%%%%% ICML 2013 EXAMPLE LATEX SUBMISSION FILE %%%%%%%%%%%%%%%%%
%%%%%%%%%%%%%%%%%%%%%%%%%%%%%%%%%%%%%%%%%%%%%%%%%%%%%%%%%%%%%%%%%%

% Use the following line _only_ if you're still using LaTeX 2.09.
%\documentstyle[icml2013,epsf,natbib]{article}
% If you rely on Latex2e packages, like most moden people use this:
\documentclass[11pt]{article}

% For figures
\usepackage{graphicx} % more modern
%\usepackage{epsfig} % less modern
\usepackage{subfigure}


% For math symbol
\usepackage{amsmath,amsfonts,amssymb}

% For theorem
\usepackage{amsthm}

% For citations
\usepackage{natbib}

% For algorithms
\usepackage{algorithm}
\usepackage{algorithmic}

% For appendix
% \usepackage[title,titletoc,toc]{appendix}

% For graphical model
% \usepackage{tikz}
% \usetikzlibrary{fit,positioning}
% \usetikzlibrary{arrows,shapes}
% As of 2011, we use the hyperref package to produce hyperlinks in the
% resulting PDF.  If this breaks your system, please commend out the
% following usepackage line and replace \usepackage{icml2013} with
% \usepackage[nohyperref]{icml2013} above.
\usepackage{hyperref}

% Packages hyperref and algorithmic misbehave sometimes.  We can fix
% this with the following command.
\newcommand{\theHalgorithm}{\arabic{algorithm}}
\newcommand{\trans}{\top}
\newcommand{\bm}{\mathbf}

\newcommand{\comment}[1]{}

% theorem macro
\newtheorem{thm}{Theorem}
\newtheorem{lem}[thm]{Lemma}

% Employ the following version of the ``usepackage'' statement for
% submitting the draft version of the paper for review.  This will set
% the note in the first column to ``Under review.  Do not distribute.''
% \usepackage{icml2013}
% \usepackage[accepted]{icml2013}
% Employ this version of the ``usepackage'' statement after the paper has
% been accepted, when creating the final version.  This will set the
% note in the first column to ``Proceedings of the...''
% \usepackage[accepted]{icml2013}

\usepackage{fullpage}
\usepackage{xcolor}
% The \icmltitle you define below is probably too long as a header.
% Therefore, a short form for the running title is supplied here:
% \icmltitlerunning{Submission and Formatting Instructions for ICML 2013}


\title{Experiments}

\begin{document}
\maketitle

%\twocolumn[
%\icmltitle{}

% It is OKAY to include author information, even for blind
% submissions: the style file will automatically remove it for you
% unless you've provided the [accepted] option to the icml2013
% package.

% \icmlauthor{Your Name}{email@yourdomain.edu}
% \icmladdress{Your Fantastic Institute,
%            314159 Pi St., Palo Alto, CA 94306 USA}
%\icmlauthor{Your CoAuthor's Name}{email@coauthordomain.edu}
%\icmladdress{Their Fantastic Institute,
%            27182 Exp St., Toronto, ON M6H 2T1 CANADA}

% You may provide any keywords that you
% find helpful for describing your paper; these are used to populate
% the "keywords" metadata in the PDF but will not be shown in the document

%\icmlkeywords{exponential family harmonium, markov random fields, Deep Boltzmann Machine, Gaussian Integral trick}

%\vskip 0.3in
%]

%\section{Multiview Latent Variable Models}

\section{Flow Cytometry Dataset}
Flow cytometry (FCM) data are multivariate measurements from flow cytometers that record light scatter and fluorescence emission properties of hundreds of thousands of individual cells. They are important to studying the cell structures of normal and abnormal cells and diagnosing human disease \cite{cytometry_nature}. This is a challenging dataset for clustering because the distribution of the data is non-Gaussian and heavily skewed.

We experimented on DLBCL dataset from the FlowCAP challenge \cite{cytometry_nature}. It contains 30 samples, and each sample consists of tens of thousands of cells measurements in 5 dimensions, with 2 to 4 clusters. Each sample is a separate clustering task, and the performance is evaluated by the weighted f-score used in \cite{cytometry_nature}. For each sample, we selected the best kernel bandwidth by log-likelihood with 5-fold cross validation. For comparison, we also evaluated the performance of EM algorithm for mixture of Gaussians with diagonal covariances, flowMeans, FLAME.



\bibliography{mlv_kernel}
\bibliographystyle{icml2013}

\end{document}
