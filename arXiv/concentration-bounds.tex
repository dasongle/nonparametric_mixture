\section{Robust Tensor Power Method}
We recap the robust tensor power method for finding the tensor eigen-pairs in Algorithm~\ref{alg:robustpower}, analyzed in detail in~\cite{AnandkumarEtal:community12} and~\cite{AnandkumarEtal:tensor12}.
The method computes the eigenvectors of a   tensor through deflation, using a set of initialization vectors. Here, we employ random initialization vectors. This can be replaced with better initialization vectors, in certain settings, e.g. in the community model, the neighborhood vectors provide better initialization and lead to stronger guarantees~\cite{AnandkumarEtal:community12}.
Given the initialization vector, the method then runs a tensor power update, and runs for $N$ iterations to obtain an eigenvector. The successive eigenvectors are obtained via deflation.

\begin{algorithm}
\caption{$\{\lambda, \Phi\}\leftarrow $TensorEigen$(T,\, \{v_i\}_{i\in [L]}, N)$}\label{alg:robustpower}
\begin{algorithmic}
\renewcommand{\algorithmicrequire}{\textbf{Input: }}
\renewcommand{\algorithmicensure}{\textbf{Output: }}
\REQUIRE Tensor $T\in \R^{k \times k \times k}$, set of $L$ initialization vectors $\{v_i\}_{i\in L}$, number of
iterations  $N$.
%\ENSURE Eigenpairs:  $\lambda$ is the vector of eigenvalues and $\Phi$ is the matrix of eigenvectors of  $T$.
\ENSURE the estimated eigenvalue/eigenvector pairs $\{\lambda, \Phi\}$, where $\lambda$ is the vector of eigenvalues and $\Phi$ is the matrix of eigenvectors.

\FOR{$i =1$ to $k$}
\FOR{$\tau = 1$ to $L$}
\STATE $\th{0}\leftarrow v_\tau$.
\FOR{$t = 1$ to $N$}
\STATE $\tilde{T}\leftarrow T$.
\FOR{$j=1$ to $i-1$ (when $i>1$)}
\IF{$|\lambda_j \inner{\th{t}^{(\tau)}, \phi_j}|>\xi$}
\STATE $\tilde{T}\leftarrow \tilde{T}- \lambda_j \phi_j^{\otimes 3}$.
\ENDIF
\ENDFOR

\STATE Compute power iteration update
$
\th{t}^{(\tau)}  :=
\frac{\tilde{T}(I, \th{t-1}^{(\tau)}, \th{t-1}^{(\tau)})}
{\|\tilde{T}(I, \th{t-1}^{(\tau)}, \th{t-1}^{(\tau)})\|}
$\ENDFOR
\ENDFOR

\STATE Let $\tau^* := \arg\max_{\tau \in L} \{ \tilde{T}(\th{N}^{(\tau)},
\th{N}^{(\tau)}, \th{N}^{(\tau)}) \}$.

\STATE Do $N$ power iteration updates starting from
$\th{N}^{(\tau^*)}$ to obtain eigenvector estimate $\phi_i$, and set $\lambda_i :=
\tilde{T}(\phi_i, \phi_i, \phi_i)$.

\ENDFOR
\RETURN the estimated eigenvalue/eigenvectors
$(\lambda, \Phi)$.

\end{algorithmic}
\end{algorithm}
%
%\section{Proof of Sample Complexity Guarantees}
%
%Let $\rho:=\sup_{x \in \Omega} k(x,x)$,   $\| \cdot\|_{}$ be the Hilbert-Schmidt norm, $\rho_{\min}:=\min_{i\in [k]} \pi_i$ and $\sigma_k(\Ccal_{X_1, X_2})$ be the $k$-th singular value of $\Ccal_{X_1, X_2}$.
%
%\begin{theorem}[Sample Bounds]\label{thm:samplebound}
%Pick  any $\delta\in (0,1)$. When the number of samples $m$ satisfies
%\beq m >\frac{\rho\rho^2  \log\frac{\delta}{2}}{\sigma_k(\Ccal_{X_1, X_2})},
%\quad \rho:= \max\left(\frac{512  }{\sigma^3_k(\Ccal_{X_1, X_2})}, \frac{C' k^2 }{\sigma_k^{3}(\Ccal_{X_1, X_2})},\frac{C^{''} k^{\frac{2}{3}} }{\pi_{\min}^{\frac{1}{3}} }\right),\eeq for some constants $C', C^{''}>0$, and the number of iterations $N$  and  the number of random initialization vectors $L$  (drawn uniformly on the sphere $\mathcal{S}^{k-1}$)  satisfy
%\begin{align*}
%N \geq C_2 \cdot \biggl( \log(k) + \log\log\Bigl(
%\frac{1}{\sqrt{\pi}_{\min}\epsilon_T} \Bigr) \biggr),\\
%\sqrt{\frac{\ln(L/\log_2(k/\delta))}{\ln(k)}}
%\cdot \Biggl( 1 - \frac{\ln(\ln(L/\log_2(k/\delta))) +
%C_3}{4\ln(L/\log_2(k/\delta))} -
%\sqrt{\frac{\ln(8)}{\ln(L/\log_2(k/\delta))}} \Biggr)
%\geq 1.02 \Biggl( 1 + \sqrt{\frac{\ln(4)}{\ln(k)}}
%\Biggr)
%.
%\end{align*}
%for constants $C_2,C_3>0$. (Note that the condition on $L$ holds with $L = \poly(k) \log(1/\delta)$.) The robust power method in~\cite{AnandkumarEtal:community12} yields eigen-pairs $(\h{\lambda}_i, \h{\phi}_i)$ such that there exists a permutation $\eta$, with probability $1-4\delta$, we have
%\[
%\|\pi^{-1/2}_{j} \mu_{X|h=j}-\h{\phi}_{\eta(j)}\| \leq 8 \epsilon_T \cdot\pi^{-1/2}_{j}
%, \qquad
%|\pi^{-1/2}_{j}-\h{\lambda}_{\eta(j)}| \leq  5\epsilon_T, \quad \forall j \in [k]
%,
%\]
%and
%\[
%\biggl\|
%T - \sum_{j=1}^k \hat\lambda_j \hat{\phi}_j^{\otimes 3}
%\biggr\| \leq 55\eps_T,
%\] where $\eps_T$ is the tensor perturbation bound
%\[ \eps_T := \|\h{\Tcal} - \Tcal\| \leq
%\frac{12 \rho \sqrt{\log\frac{\delta}{2}}}{\sqrt{m} \, \sigma_k^{1.5}(\Ccal_{X_1, X_2})} + \frac{512 \sqrt{2} \rho^3 \left(\log\frac{\delta}{2}\right)^{1.5}}{m^{1.5} \,\sigma_k^{3}(\Ccal_{X_1, X_2}) \sqrt{\pi}_{\min}}\]
%
%\end{theorem}
%The above result provides bounds on the estimated eigen-pairs using the robust tensor power method.
%The proof is in Appendix~\ref{app:samplebound}.

\section{Proof of Theorem~\ref{thm:samplebound}}\label{app:samplebound}

\subsection{Recap of Perturbation Bounds for the Tensor Power Method}

We now recap the result of~\citet[Thm. 13]{AnandkumarEtal:community12} that establishes bounds on the eigen-estimates under good initialization vectors for the above procedure.
Let $\Tcal=\sum_{i\in [k]}\lambda_i v_i$, where $v_i$ are orthonormal vectors and $\lambda_1\geq \lambda_2\geq\ldots \lambda_k$. Let $\h{\Tcal}=\Tcal+E$ be the perturbed tensor with $\|E\|\leq \epsilon_{T}$. Recall that $N$ denotes the number of iterations of the tensor power method.
We call an initialization vector $u$ to be $(\gamma, R_0)$-good  if there exists $v_i$ such that $\inner{u}{v_i}> R_0$
  and $|\inner{u}{v_i}| -\max_{j<i} |\inner{u}{v_j}| > \gamma  |\inner{u}{v_i}|$.   Choose $\gamma=1/100$.


\begin{theorem}
\label{thm:robustpower}
There exists universal constants $C_1, C_2 > 0$  such that the
following holds.
\beq\label{eqn:robustpowerconditions}
\epsilon_{T} \leq C_1 \cdot \lambda_{\min} R_0^2,
\qquad
N \geq C_2 \cdot \left( \log(k) + \log\log\left(
\frac{\lambdamax}{\epsilon_T} \right) \right)
,
\eeq Assume there is at least one good initialization vector corresponding to each $v_i$, $i\in [k]$. The parameter $\xi$ for choosing deflation vectors in each iteration of the tensor power method in Procedure~\ref{alg:robustpower}  is chosen as $\xi\geq 25 \eps_T$. We obtain  eigenvalue-eigenvector pairs  $(\hat\lambda_1,\hat{v}_1), (\hat\lambda_2,\hat{v}_2), \dotsc,
(\hat\lambda_k,\hat{v}_k)$ such that  there exists a permutation $\eta$ on
$[k]$ with
\[
\|v_{\eta(j)}-\hat{v}_j\| \leq 8 \epsilon_T/\lambda_{\eta(j)}
, \qquad
|\lambda_{\eta(j)}-\hat\lambda_j| \leq 5\epsilon_T , \quad \forall j \in [k]
,
\]
and
\[
\left\|
\Tcal - \sum_{j=1}^k \hat\lambda_j \hat{v}_j^{\otimes 3}
\right\| \leq 55\eps_T .
\]
\end{theorem}

In the sequel, we establish concentration bounds that allows us to translate the above condition on tensor perturbation~\eqref{eqn:robustpowerconditions}  to sample complexity bounds.

\subsection{Concentration Bounds}

\subsubsection{Analysis of Whitening}

Recall that we use the covariance operator $\Ccal_{X_1 X_2}$ for whitening the 3rd order embedding $\Ccal_{X_1, X_2, X_3}$. We first analyze the perturbation in whitening when sample estimates are employed.

Let $\h{\Ccal}_{X_1 X_2}$ denote the sample covariance operator between variables $X_1$ and $X_2$, and let \[B:=0.5(\h{\Ccal}_{X_1 X_2}+ \h{\Ccal}_{X_1 X_2}^\top)=\h{\Ucal}\h{S}\h{\Ucal}^\top\] denote the SVD.
Let $\h{\Ucal}_k$ and $\h{S}_k$ denote the restriction to top-$k$ eigen-pairs, and let $B_{k} := \h{\Ucal}_k \h{S}_k \h{\Ucal}_k^\top$. Recall that the whitening matrix is given by $\h{\Wcal}:=\h{\Ucal}_k \h{S}_k^{-1/2}$. Now $\h{\Wcal}$ whitens $B_k$, i.e. $\h{\Wcal}^\top B_{k} \h{\Wcal}=I$.

Now consider the SVD of
\[ \h{\Wcal}^\top \Ccal_{X_1 X_2} \h{\Wcal}= A D A^\top,\] and define \[\Wcal:= \h{\Wcal} AD^{-1/2}A^\top, \] and $\Wcal$ whitens $\Ccal_{X_1 X_2}$ since $\Wcal^\top  \Ccal_{X_1 X_2} W=I$.
Recall that by exchangeability assumption,
\beq\label{eqn:pairsexpression} \Ccal_{X_1,X_{2}}
  = \sum_{j=1}^k \pi_j \cdot \mu_{X|j} \otimes \mu_{X|j} \iffalse+  E_{X_1 X_2}\fi= M \Diag(\pi) M^\top \iffalse +E_{X_1 X_2}\fi, \eeq where the $j^{\tha}$ column of $M$, $M_j = \mu_{X|j}$.

We now establish the following perturbation bound on the whitening procedure. Recall from \eqref{eqn:deltapairs}, $ \epsilon_{pairs}:=\nbr{\Ccal_{X_1,X_{2}} - \widehat \Ccal_{X_1,X_{2}}}_{}$. Let $\sigma_1(\cdot) \geq \sigma_2(\cdot)\ldots$ denote the singular values of an operator.

\begin{lemma}[Whitening perturbation]\label{lemma:whiten} Assuming that $\epsilon_{pairs} < 0.5 \sigma_k(\Ccal_{X_1 X_2})$,
\beq \epsilon_{W}:= \|\Diag(\pi)^{1/2}M^\top(\h{\Wcal}-\Wcal)\|\leq \frac{4\epsilon_{pairs} \iffalse+2\sigma_{k+1}(\Ccal_{X_1 X_2})\fi}{ \sigma_{k}(\Ccal_{X_1 X_2})}\iffalse\cdot (1+ \sigma_{k+1}(\Ccal_{X_1 X_2}))\fi\eeq
\end{lemma}

\paragraph{Remark: }Note that $\sigma_{k}(\Ccal_{X_1 X_2}) = \sigma_{k}^2(M)$.

\bprf The proof is along the lines of Lemma 16 of~\cite{AnandkumarEtal:community12}, but adapted to whitening using the covariance operator here.
 \begin{align*}\|\Diag(\pi)^{1/2} M^\top(\h{\Wcal}-\Wcal)\|&=
\|\Diag(\pi)^{1/2} M^\top W(A D^{1/2} A^\top -I)\|\\ &\leq\|\Diag(\pi)^{1/2} M^\top \Wcal\| \|D^{1/2}-I\|. \end{align*} Since $\Wcal$ whitens $\Ccal_{X_1 X_2}=M \Diag(\pi) M^\top\iffalse+E\fi$, we have that $\|\Diag(\pi)^{1/2} M^\top \Wcal\| =1$\iffalse\leq \|I-E\|^{1/2}\leq 1+ \sigma_{k+1}(\Ccal_{X_1 X_2})\fi. Now we control $\|D^{1/2}-I\|$.  Let $\tl{E}:= \Ccal_{X_1,X_{2}} -B_k$, where recall that $B=0.5( \widehat \Ccal_{X_1,X_{2}}+ \h{\Ccal}_{X_1 X_2}^\top)$ and $B_k$ is its restriction to top-$k$ singular values. Thus, we have $\|\tl{E}\| \leq \epsilon_{pairs} + \sigma_{k+1}(B)\leq 2\epsilon_{pairs}\iffalse+\sigma_{k+1}(\Ccal_{X_1 X_2})\fi$.
 We now have
\begin{align*}
\|D^{1/2}-I\|&\leq \|(D^{1/2}-I)(D^{1/2}+I)\|\leq \|D-I\|
\\ &=\|AD A^\top - I\| = \|\h{\Wcal}^\top \Ccal_{X_1 X_2}  \h{\Wcal} -I\|\\ &=\| \h{\Wcal}^\top  \tl{E} \h{\Wcal}\| \leq \|\h{\Wcal}\|^2 ( 2 \epsilon_{pairs}\iffalse+\sigma_{k+1}(\Ccal_{X_1 X_2})\fi).
\end{align*}Now
\[ \|\h{\Wcal}^2\| \leq\frac{1}{ \sigma_k(\h{\Ccal}_{X_1 X_2})}\leq \frac{2}{\sigma_k(\Ccal_{X_1 X_2})},\] when  $\epsilon_{pairs}<0.5 \sigma_k(\Ccal_{X_1 X_2})$.
\eprf

\subsubsection{Tensor Concentration Bounds}

Recall that the whitened tensor from samples is given by
$$\h{\Tcal} := \h{\Ccal}_{X_1 X_2 X_3} \times_1 (\h{\Wcal}^\top) \times_2 (\h{\Wcal}^\top) \times_3 (\h{\Wcal}^\top).$$ We want to establish its perturbation from the whitened tensor using exact statistics
$$\Tcal := \Ccal_{X_1 X_2 X_3} \times_1 (\Wcal^\top ) \times_2 (\Wcal^\top ) \times_3 (\Wcal^\top ).$$ Further, we have
\beq\label{eqn:triplesexpression}\Ccal_{X_1 X_2 X_3}= \sum_{h\in [k]} \pi_h \cdot \mu_{X|h} \otimes \mu_{X|h} \otimes \mu_{X|h} \iffalse+ E_{X_1, X_2, X_3}\fi\eeq

Let $\epsilon_{triples}:= \|\h{\Ccal}_{X_1 X_2 X_3}-\Ccal_{X_1 X_2 X_3}\|_{}$. Let $\pi_{\min}:=\min_{h\in [k]}\pi_h$.

\begin{lemma}[Tensor perturbation bound]
Assuming that $\epsilon_{pairs} < 0.5 \sigma_k(\Ccal_{X_1 X_2})$, we have
\beq\label{eqn:epsilonT} \epsilon_T:= \|\h{\Tcal} - \Tcal\|
\leq \frac{2\sqrt{2}\epsilon_{triples}}{\sigma_k(\Ccal_{X_1 X_2})^{1.5} }+\frac{\epsilon_W^3}{\sqrt{\pi_{\min}}}\iffalse+ \|E_{X_1 X_2 X_3}\| \frac{\epsilon_W^3}{\pi_{\min}^{1.5} \sigma_k(M)^3} \fi.\eeq
\end{lemma}


\bprf  Define
  intermediate tensor
\begin{align*} \tl{\Tcal}&:= \Ccal_{X_1 X_2 X_3} \times_1 (\h{\Wcal}^\top) \times_2 (\h{\Wcal}^\top) \times_3 (\h{\Wcal}^\top).\end{align*}
We will bound $\|\h{\Tcal}-\tl{\Tcal}\|$  and $\| \h{\Tcal}-\Tcal\|$  separately.
\begin{align*}
\|\h{\Tcal}-\tl{\Tcal}\| &\leq \|\h{\Ccal}_{X_1, X_2, X_2} - \Ccal_{X_1, X_2, X_3}\| \|\h{\Wcal}\|^3\leq \frac{2\sqrt{2}\epsilon_{triples}}{\sigma_k(\Ccal_{X_1 X_2})^{1.5} },
\end{align*}using the bound on $\|\h{\Wcal}\|$ in Lemma~\ref{lemma:whiten}. For the other term,
first note that
\[ \Ccal_{X_1, X_2, X_3} = \sum_{h\in [k]} \pi_h \cdot M_h \otimes M_h \otimes M_h \iffalse+ E_{X_1, X_2, X_3}\fi, \]
 \iffalse where $\|E_{X_1, X_2, X_3}\|$ is the residual and we need to bound this in non-parametric case.\fi
\begin{align*} \|\h{\Tcal}-\Tcal\|&= \| \Ccal_{X_1 X_2 X_3}\times_1 (\h{\Wcal} -\Wcal)^\top \times_2 (\h{\Wcal} -\Wcal)^\top \times_3 (\h{\Wcal}-\Wcal)^\top\| \\
&\leq \frac{ \| \Diag(\pi)^{1/2}M^\top(\h{\Wcal}-\Wcal)\|^3}{\sqrt{\pi_{\min}}}
\iffalse+ \|E_{X_1 X_2 X_3}\| \|\h{\Wcal}-\Wcal\|^3\fi\\
&= \frac{\epsilon_W^3}{\sqrt{\pi_{\min}}}
\iffalse+ \|E_{X_1 X_2 X_3}\| \frac{\epsilon_W^3}{\pi_{\min}^{1.5} \sigma_k(M)^3}\fi
\end{align*}
\eprf\\



\bprfof{Theorem~\ref{thm:samplebound}}
We obtain a condition on the above perturbation $\epsilon_T$ in \eqref{eqn:epsilonT} by applying Theorem~\ref{thm:robustpower} as
$ \epsilon_T\leq C_1\lambda_{\min} R_0^2$. Here, we have $\lambda_{i} = 1/\sqrt{\pi_{i}}\geq 1$. For random initialization, we have that $R_0 \sim 1/\sqrt{k}$, with probability $1-\delta$ using $\poly(k) \poly(1/\delta)$ trials, see Thm. 5.1 in~\cite{AnandkumarEtal:tensor12}. Thus, we require that $ \epsilon_T  \leq \frac{C_1}{k}$. Summarizing, we require for the following conditions to hold
\beq\epsilon_{pairs}\leq 0.5 \sigma_k(\Ccal_{X_1 X_2}), \quad \epsilon_T  \leq \frac{C_1}{k}.\eeq
We now substitute for $\epsilon_{pairs}$ and $\epsilon_{triples}$ in \eqref{eqn:epsilonT} using Lemma~\ref{lemma:pairs} and Lemma~\ref{lemma:triples}.

%First consider the case when   $H$ is exactly a $k$-way categorical variable.

%\paragraph{Case of $k$-component mixture: }Here, $E_{X_1, X_2}=0$ and $E_{X_1,X_2,X_3}=0$ in \eqref{eqn:pairsexpression} and \eqref{eqn:triplesexpression}.

From Lemma~\ref{lemma:pairs}, we have that
\[ \epsilon_{pairs}  \leqslant \frac{2\sqrt{2}\rho \sqrt{\log\frac{\delta}{2}}}{\sqrt{m}}, \]with probability $1-\delta$. It is required that $\epsilon_{pairs} < 0.5 \sigma_k(\Ccal_{X_1, X_2})$, which yields that \beq\label{eqn:cond1} m > \frac{32 \rho^2 \log\frac{\delta}{2}}{\sigma^2_k(\Ccal_{X_1, X_2})}.\eeq
Further we require that $\epsilon_T \leq C_1/k$, which implies that each of the terms in \eqref{eqn:epsilonT} is less than $C/k$, for some constant $C$. Thus, we have
\[ \frac{2\sqrt{2} \epsilon_{triples}}{\sigma_k^{1.5}(\Ccal_{X_1, X_2})} < \frac{C}{k}\quad\Rightarrow\quad m > \frac{C_3 k^2 \rho^3 \log \frac{\delta}{2}}{\sigma_k^{3}(\Ccal_{X_1, X_2})},\]
for some constant $C_3$ with probability $1-\delta$ from Lemma~\ref{lemma:triples}. Similarly for the second term in \eqref{eqn:epsilonT}, we have
\[\frac{\epsilon_W^3}{\sqrt{\pi_{\min}}}< \frac{C}{k},\]and from Lemma~\ref{lemma:whiten}, this implies that \[ \epsilon_{pairs} \leq \frac{C' \pi_{\min}^{1/6} \sigma_k(\Ccal_{X_1, X_2})}{k^{1/3}\iffalse(1+\sigma_{k+1}(\Ccal_{X_1, X_2}))\fi},\]Thus, we have
\[  m > \frac{C_4 k^{\frac{2}{3}} \rho^2 \log\frac{\delta}{2}\iffalse (1+\sigma_{k+1}(\Ccal_{X_1, X_2}))^2\fi}{\pi_{\min}^{\frac{1}{3}} \sigma^2_k(\Ccal_{X_1, X_2})}, \]for some other constant $C_4$ with probability $1-\delta$. \iffalse Additionally, we also require
\[ (1+\sigma_{k+1}(\Ccal_{X_1, X_2}))\sigma_{k+1}(\Ccal_{X_1, X_2})\leq \frac{C_5 \pi_{\min}^{1/6} \sigma_k(\Ccal_{X_1, X_2})}{k^{1/3}}.\]\fi
 Thus, we have the result in Theorem~\ref{thm:samplebound}.


\eprfof

\subsubsection{Concentration bounds for Empirical Operators}

Concentration results for the singular value decomposition of empirical operators.

\begin{lemma}[Concentration bounds for pairs]\label{lemma:pairs} Let $\rho:=\sup_{x \in \Omega} k(x,x)$, and $\| \cdot\|_{}$ be the Hilbert-Schmidt norm, we have for \beq \epsilon_{pairs}:=\nbr{\Ccal_{X_1 X_2} - \widehat \Ccal_{X_1 X_2}}_{},\label{eqn:deltapairs} \eeq
\begin{eqnarray}
	\Pr \cbr{\epsilon_{pairs}  \leqslant \frac{2\sqrt{2}\rho \sqrt{\log\frac{\delta}{2}}}{\sqrt{m}} } \geqslant 1-\delta. \label{eq:operator_concentration}
\end{eqnarray}
\end{lemma}

\begin{proof}
We will use similar arguments as in~\cite{RosBelVit2010} which deals with symmetric operator. Let $\xi_{i}$ be defined as
\begin{eqnarray}
\xi_{i}\, =\, \phi(x_1^i) \otimes \phi(x_2^i) - \Ccal_{X_1,X_2}.
\end{eqnarray}
It is easy to see that $\mathbb{E}[\xi_{i}] = 0$. Further, we have
\begin{eqnarray}
	\sup_{x_1,x_2} \nbr{\phi(x_1) \otimes \phi(x_2)}^{2}_{}
    = \sup_{x_1,x_2} k(x_1, x_1) k(x_2,x_2)
    \leqslant \rho^{2},
\end{eqnarray}
which implies that $\nbr{\Ccal_{X_1 X_2}}_{} \leqslant \rho$, and $\nbr{\xi_i}_{} \leqslant 2 \rho$. The result then follows from the Hoeffding's inequality in Hilbert space.
\end{proof}

Similarly, we have the concentration bound for 3rd order embedding.

\begin{lemma}[Concentration bounds for triples]\label{lemma:triples} Let $\rho:=\sup_{x \in \Omega} k(x,x)$, and $\| \cdot\|_{}$ be the Hilbert-Schmidt norm, we have for \beq \epsilon_{triples}:=\nbr{\Ccal_{X_1 X_2 X_3} - \widehat \Ccal_{X_1 X_2 X_3}}_{},\label{eqn:deltapairs} \eeq
\begin{eqnarray}
	\Pr \cbr{\epsilon_{triples}  \leqslant \frac{2\sqrt{2}\rho^{3/2} \sqrt{\log\frac{\delta}{2}}}{\sqrt{m}} } \geqslant 1-\delta. \label{eq:operator_concentration2}
\end{eqnarray}
\end{lemma}

\begin{proof}
We will use similar arguments as in~\cite{RosBelVit2010} which deals with symmetric operator. Let $\xi_{i}$ be defined as
\begin{eqnarray}
\xi_{i}\, =\, \phi(x_1^i) \otimes \phi(x_2^i) \otimes \phi(x_3^i)  - \Ccal_{X_1 X_2 X_3}.
\end{eqnarray}
It is easy to see that $\mathbb{E}[\xi_{i}] = 0$. Further, we have
\begin{eqnarray}
	\sup_{x_1,x_2,x_3} \nbr{\phi(x_1) \otimes \phi(x_2) \otimes \phi(x_3)}^{2}_{}
    = \sup_{x_1,x_2,x_3} k(x_1, x_1) k(x_2,x_2) k(x_3,x_3)
    \leqslant \rho^{3},
\end{eqnarray}
which implies that $\nbr{\Ccal_{X_1 X_2 X_3}}_{} \leqslant \rho^{3/2}$, and $\nbr{\xi_i}_{} \leqslant 2 \rho^{3/2}$. The result then follows from the Hoeffding's inequality in Hilbert space.
\end{proof}
